% !TEX root = ../Dissertation.tex
%===================================================================================================

\chapter{Results}




\section{NOD mice have a significantly increased population of thymic B cells}

As mentioned previously in \cref{sec:thymicBcells}, it is normal for a small population of B cells, about 0.1-0.5\% of thymocytes, to be present in the thymus of both humans and mice \citep{Perera2013}.
However, it has been noted by our lab and others \citep{OReilly1994, Serreze1998}, that this population is significantly increased in the NOD mouse.
It is thought that these cells may have a role in the process of T cell negative selection \citep{Starr2003, Perera2013}.
It is not known how thymic B cells come to populate the thymus, however, it is proposed that they either migrate there from the bone marrow, or that they develop within the thymus from progenitors there which have B cell potential \citep{Perera2013}.

To confirm the increase in thymic B cells in NOD mice compared to B6 controls, a time-course study was conducted to assess the presence of mature B cells in the thymi of NOD mice and B6 control mice that do not spontaneously develop T1D.
Time points for analysis were chosen to correlate with different stages of the disease process, based on the well defined NOD mouse disease progression (see \cref{fig:diseasecourse}).
As shown in \cref{subfig:IncthyBcells}, as NOD mice age and progress along the T1D development pathway, the frequency of mature thymic B cells increases significantly in comparison to B6 controls. (Varian and Green, unpublished observations).

To see whether this increase in thymic B cells is having an effect on T cell development in the thymus, the frequencies of DN, DP and SP T cells in NOD and B6 thymi were compared using flow cytometry.
Analysis of this data, shown in \cref{fig:NODB6Tcells}, revealed that there was no significant difference between NOD and B6 developing T cell populations at any age.
%This would give the impression that the B cell abnormality does not result from a defect in the whole thymus and is more likely to be B cell specific.


It was also necessary to confirm whether or not the NOD mouse has an increased population of B cells overall which could account for the increase in the thymus.
It is possible that if NOD mice are producing more B cells in the bone marrow as they age, there may be increased migration of B cells to secondary lymphoid organs such as the spleen and lymph nodes, as well as the thymus.
To assess this possibility, a time-course study was carried out to look at the frequencies of B cells in the bone marrow of NOD and B6 mice at the same time points as the thymus time-course study.
As shown in \cref{subfig:BMBcells}, there is no significant difference between the frequency of B cells in the NOD and B6 bone marrow at any age, whereas the frequency of B cells is significantly increased in the thymus at both 9 and 13 weeks of age.
Similarly, no significant difference in B cell frequencies were seen in either spleen or lymph nodes of NOD and B6 mice (Data not shown).
This would suggest that the increase in B cells is specific to the thymus, not to the mouse as a whole.

\begin{figure}
	\begin{subfigure}{0.5\textwidth}
	\includegraphics[width=\textwidth]{Figures/IncthyBcells.pdf}
	\caption{}
	\label{subfig:IncthyBcells}
	\end{subfigure}
	\begin{subfigure}{0.5\textwidth}
	\includegraphics[width=\textwidth]{Figures/ThyBcellnumbers.pdf}
	\caption{}
	\label{subfig:ThyBcellnumbers}
	\end{subfigure}
	\begin{subfigure}{0.5\textwidth}
	\includegraphics[width=\textwidth]{Figures/BMBcells.pdf}
	\caption{}
	\label{subfig:BMBcells}
	\end{subfigure}
\caption[NOD thymi have significantly more B cells than B6 thymi]{Frequency (a) and absolute number (a) of B cells is significantly increased in the NOD thymus, but not bone marrow (c) compared to B6 controls.
Data gated lymphocytes, single cells then mature B cells were identified based on the expression of CD19 and IgM.
The Mann Whitney test was used to compare the difference in mature B cell frequency at each age. Differences between NOD and B6 were statistically significant at both 9 and 12 weeks of age (P=0.0190 and P=0.0040, respectively). The difference at 4 weeks is not statistically significant.
NOD n=3-4, B6 n=3-6.}
\label{fig:IncthyBcells}
\end{figure}

\begin{figure}
	\begin{subfigure}{0.5\textwidth}
	\includegraphics[width=\textwidth]{Figures/4wkThyTcells.pdf}
	\caption{}
	\label{subfig:4wkThyTcells}
	\end{subfigure}
	\begin{subfigure}{0.5\textwidth}
 	\includegraphics[width=\textwidth]{Figures/9wkThyTcells.pdf}
	\caption{}
	\label{subfig:9wkThyTcells}
	\end{subfigure}
	\begin{subfigure}{\textwidth}
	\centering
 	\includegraphics[width=0.5\textwidth]{Figures/12wkThyTcells.pdf}
	\caption{}
	\label{subfig:12wkThyTcells}
	\end{subfigure}
\caption[T cell development is normal in the NOD mouse]{T cell populations are normal between NOD and B6. (a) 4 weeks, (b) 9 weeks, (c) 12 weeks.}
\label{fig:NODB6Tcells}
\end{figure}

The mechanism by which B cell frequencies/numbers increase in the NOD thymus is unknown. 
Two potential hypotheses to account for the increase are development from progenitors entering the thymus which have B cell potential, or migration of B cells from the bone marrow. \toref{}
he current consensus is that B cells are developing within the thymus. 


\subsection{There are pro and pre B cells present in the NOD thymus}
\label{subsec:proandpre}


B cells in the bone marrow develop through different developmental stages (See \cref{subsubsec:committedBcelldevelopment}).
The first of these stages is the CD19\textsuperscript{+}CD43\textsuperscript{+}IgM\textsuperscript{-} pro B cell stage.
Subsequently, the IgM heavy chain is rearranged and coupled to a surrogate light chain (such as VPreB), forming the preBCR. \fig{Figure showing b cell development}
The sucessful formation of a preBCR allows progression to the CD19\textsuperscript{-}CD43\textsuperscript{-}IgM\textsuperscript{-} pre B cell stage.
Next, the light chain rearranges resulting in a full IgM molecule and the B cell can be released from the bone marrow to migrate to the spleen to finish its development.

%Pro and pre B cells are seen in normal B cell development in the bone marrow of both NOD and B6 mice, as shown in \cref{subfig:BMpropre}.
%The similarity in the bone marrow between the two strains suggests that B cell development in the bone marrow of the NOD mouse is normal.

\todo{make sure figure actually shows NOD and B6!!}
\citet{Akashi2000} has shown pro and pre B cells in the thymus of B6 mice. 
To confirm this, and to see if they are present in the NOD mouse, a comparative flow cytometric study was carried out using NOD and B6 mouse thymi.
As a control, the populations of pro and pre B cells were also assessed as this is the normal site of B cell development.



As shown in \cref{subfig:Thypropre}, pro and pre B cells can be seen in the thymus of NOD and B6 mice.
Data shown is representative of 3-6 mice aged between 6-9 weeks. %Thymus data HLD 20th July, BM data JV 23.9.09
This age was chosen as it is the time when mature B cells increase in the thymus of NOD mice compared to B6 mice (\cref{subfig:IncthyBcells}).
The presence of pro and pre B cells suggests that these stages of B cell development may be supported by the thymus. 
%however, it is not possible to say whether these cells have developed in the thymus or migrated there.
%It can only suggest that the thymus is capable of providing an environment where they are able to survive.
As expected, the bone marrow (\cref{subfig:BMpropre}) also shows populations of pro and pre B cells which appear similar between mouse strains.


%Presence of these developing B cells suggest that B cells could be developing within the thymus and that further investigation is warranted.}



\begin{figure}	
	\begin{subfigure}{\textwidth}
	\centering
	\includegraphics[width=0.7\textwidth]{Figures/Thymuspropre.png}
	\caption{Thymus}
	\label{subfig:Thypropre}
	\end{subfigure}
	\begin{subfigure}{\textwidth}
	\centering
	\includegraphics[width=0.7\textwidth]{Figures/Bonemarrowpropre.png}
	\caption{Bone marrow}
	\label{subfig:BMpropre}
	\end{subfigure}
\caption[Pro and pre B cells are present in NOD and B6 thymi]{Pro and pre B cells are found in the bone marrow and thymus of both NOD and B6. Single cell suspensions prepared and stained with CD19, IgM and CD43 antibodies for flow cytometry. Thymic samples gated on acquisition for CD19\textsuperscript{+} cells (100\% CD19\textsuperscript{+} cells, approx. 2\% everything else). For analysis, samples were gated lymphocytes, single cells, CD19\textsuperscript{+}, IgM\textsuperscript{-} then split into pro and pre B cells using CD43 expression. Bone marrow plots are representative of 4 NOD and 6 B6 mice. Thymus plots are representative of 3 NOD and 4 B6 mice. All mice are aged between 6-9 weeks of age.}
\label{fig:PropreBcells}
\end{figure}





\subsection{B cell development may be dependent on the presence of a mature B cell}

In the bone marrow, progression from the pro B cell stage to the pre B cell stage requires rearrangement of the IgM heavy chains.
Ablation of IgM heavy chain rearrangement results in an accumulation of pro B cells in the bone marrow \toref{}.
To confirm that the requirement for rearrangement in B cell development was the same in the thymus as the bone marrow, comparative flow cytometric analysis of NOD and NOD KO thymi was carried out to investigate developing B cell populations.
%Following on from the finding of pro and pre B cells in NOD and B6 thymi, it was wondered if the frequency of pro B cells is different between the strains of mice.
%In particular, thought was given to whether the frequency of pro B cells would be increased in NOD mice compared to B6 which could account for the increased B cells seen in the NOD thymi.
%NOD KO mice were also investigated to see how they compared to the NOD and B6.  

NOD KO mice are unable to rearrange their IgM heavy chain and therefore cannot progress beyond the pro B cell stage (See \cref{methods:mice}). 
Comparative, flow cytometric studies of NOD and NOD KO bone marrow, as expected, revealed that in NOD KO mice, B cell development was blocked at the pro B cell stage \cref{subfig:KOBM}.
It was therefore expected that in the NOD KO thymus there would also be a population of pro B cells, similar to that seen in the NOD WT mice. 
However, this does not appear to be the case.

When analysing this data, it was important to establish a gating system whereby the three different strains could be compared fairly.
NOD KO mice have no pre B cells and no IgM\textsuperscript{+} cells therefore previous gating to look at pro and pre B cells (CD19\textsuperscript{+} the analysed on IgM and CD43 expression) would be skewed in the NOD KO as 100\% of the cells would be pro B cells, whether or not the population was increased or decreased with respect to the other strains.
With this in mind, IgM\textsuperscript{+} cells were gated out, as were CD43\textsuperscript{-}.
This leaves only IgM\textsuperscript{-}CD43\textsuperscript{+} cells, of which any that are CD19\textsuperscript{+} should be pro B cells.
This means that the frequencies of pro B cells can be compared fairly by the frequency of CD19\textsuperscript{+} cells present within the IgM\textsuperscript{-}CD43\textsuperscript{+} population.

As shown in \cref{subfig:MatureBincproBgraph}, the difference between the NOD and B6 is not significant suggesting that an increase in pro B cells cannot account for the increase in thymic B cells seen in the NOD thymus.
However, the frequency of pro B cells is statistically significantly decreased in the NOD KO thymi compared to both NOD WT and B6 thymi.
This is an interesting finding as it suggests that the enhancement of pro B cells in the thymus may be dependent on the presence of a mature B cell.


\begin{figure}
	\begin{subfigure}{\textwidth}
	\includegraphics[width=\textwidth]{Figures/MatureBincproB.png}
	\caption{CD19+CD43+IgM- pro B cells in, from left to right, B6, NOD and NOD KO mice of 6-8 weeks of age}
	\end{subfigure}
	\begin{subfigure}{\textwidth}
	\centering
	\includegraphics[width=\textwidth]{Figures/B6NODKOBM.png}
	\caption{NOD bone marrow shows pro, pre and IgM\textsuperscript{+} B cells whereas NOD KO bone marrow only shows pro B cells}
	\label{subfig:KOBM}
	\end{subfigure}
	\begin{subfigure}{\textwidth}
	\centering
	\includegraphics[width=0.45\textwidth]{Figures/MatureBincproBgraph.pdf}
	\caption{Graph to show difference in pro B cell frequency in NOD, NOD KO and B6 mouse thymi}
	\label{subfig:MatureBincproBgraph}
	\end{subfigure}
\caption[NOD mice have an increased frequency of pro B cells compared to NOD KO thymi]{Mature B cells increase frequency of pro B cells in the thymus.
Flow cytometric analysis of B6, NOD and NOD KO thymi was carried out.
Single cell suspensions were prepared and subsequently incubated with appropriate analytic fluorescent-labelled antibody.
Samples were gated on acquisition for CD19\textsuperscript{+} cells (100\% of CD19\textsuperscript{+} and only approx. 2\% of other cells).
Analytic gating firstly gated on lymphocytes and single cells.
Following this, IgM\textsuperscript{+} cells and CD43\textsuperscript{-} cells were gated out to make all 3 strains of mice equal as only pro B cells should be CD19\textsuperscript{+} in this population. 
Top panel shows representative FACS plots showing frequency of CD19\textsuperscript{+} cells within the IgM\textsuperscript{-}CD43\textsuperscript{-} gate, which equates to pro B cells.
The middle panel shows the populations of pro, pre and IgM\textsuperscript{+} B cells in the bone marrow of stated mice.
The bottom panel shows frequency of pro B cells in the thymi of NOD, NOD KO and B6 thymi.
Statistical significance was determined using a One-way ANOVA and Tukey's test. P=0.0025. Tukey's test showed statisticaly significance between NOD KO and NOD and NOD KO and B6 but the difference between NOD and B6 is not statistically significant.
Experiment was carried out single blind, NOD n=3, NOD KO n=4, B6 n=4.}
\label{fig:MatureBincProB}
\end{figure}


%\subsection{NOD mice express higher levels of CD43 in the thymus}
%\label{Results:CD43}

%During analysis of the data showing pro B cell presence in the NOD thymus, it was noted that there were differences in CD43 expression between mice of NOD background and B6 controls.
%CD43 is thought to be involved in T cell activation and adhesion in vitro \citep{McEvoy1997} and T cell migration from the blood into lymphoid tissues in vivo \citep{Johnson1999}.
%Further to this, it seems that an anti-CD43 antibody is capable of preventing T1D in NOD mice \citep{Johnson1999}, suggesting that it may have a role in T1D pathogenesis.
%Given the potential implication of CD43 in T1D pathogenesis, the discrepancy in CD43 expression was investigated.

%Samples were gated on lymphocytes, single cells and IgM\textsuperscript{-} cells.
%The frequency of IgM\textsuperscript{-} cells in the thymus was very similar between all mouse strains, however, CD43 expression appeared to be increased in NOD and NOD KO compared to B6, as shown in \cref{fig:CD43expression}.
%Statistical analysis shows that the difference in frequencies of IgM\textsuperscript{-}CD43\textsuperscript{+} between the NOD KO and B6 is statistically significant \todo{P values, check stats}.
%However, there appears to be a wide spread of frequencies of IgM\textsuperscript{-}CD43\textsuperscript{+} cells in both NOD and NOD KO mice.
%This suggests that the experiments need to be repeated to try and increase the reproducibility of the results.
%This spread is not seen in the B6 mouse suggesting that the thymic environment of the B6 mouse may be more regulated than the NOD which shows varying frequencies of IgM\textsuperscript{-}CD43\textsuperscript{+} cells.


%\begin{figure}
	%\begin{subfigure}{\textwidth}
	%\includegraphics[width=\textwidth]{Figures/CD43expression.png}
	%\caption{Figure showing CD43 expression in IgM- cells in the thymus}
	%\end{subfigure}
	%\begin{subfigure}{\textwidth}
	%\centering
	%\includegraphics[width=0.6\textwidth]{Figures/CD43levels.pdf}
	%\caption{Graph showing increased CD43 expression NOD}
	%\end{subfigure}
%\caption[Mouse of the NOD background have higher levels of CD43 expression compared to B6]{NOD mice show increased CD43 expression in the thymus.
%Top panel shows representative FACS plots showing increased CD43 expression in NOD WT and NOD KO mice.
%However, the graph shows that the difference is only significant between the NOD KO and B6, analysed using a One-Way ANOVA and Tukey's test.
%FACS plots gated lymphocytes, single cells, IgM\textsuperscript{-} then analysed on CD43 expression.
%NOD KO n=4, NOD WT n=3, B6 n=4. 
%Experiment carried out single blind.}
%\label{fig:CD43expression}
%\end{figure}

%CD43 levels on B cells from last year???

\subsection{Transfers}

The finding that the frequency of pro B cells was decreased in NOD KO mice with respect to NOD mice suggests that a mature B cell might have an impact on the recruitment or survival of pro B cells in the thymus of NOD mice.
To investigate this further, a pilot study was set up whereby thymic cells, including thymic B cells, were transferred from NOD mice into NOD KO mice.
The aim was to examine whether transferred NOD thymic B cells can have an effect on the endogenous thymic pro B cell population in the NOD KO recipient.

Due to the preliminary nature of the transfer experiment, very small numbers of mice were used.
In the first transfer, age-matched, sex-matched, female, 10 week old donor NOD and recipient NOD KO mice were used.
Ten week old mice were chosen because.... \todo{why choose that age}.
The experimental set up for this transfer experiment is shown in \cref{fig:KOtransfersetup}.
Two donor mice were taken and their thymi were split into CD19\textsuperscript{+} and CD19\textsuperscript{-} fractions.
The fractions were then injected intravenously into the NOD KO recipients.
Recipient tissues were analysed at 7 and 11 days post transfer (See \cref{fig:KOtransfersetup}).

Further to this transfer experiment, a control transfer experiment whereby NOD-GFP thymic cells were transferred to NOD recipients to see if the normal NOD thymus would be able to entice transferred donor thymus-derived B cells back to the thymus.
While the use of congenic mice, such as CD45.1 mice, would enable tracking of CD45.2 congenic hosts, unfortunately these mice were not available.
As an alternative, NOD-RAG-GFP mice were utilised, the reason being that CD19\textsuperscript{+}GFP\textsuperscript{+} would still be present a few days post transfer due to the slow kinetics of GFP degradation.
The experimental setup of this transfer is shown in \cref{subfig:GFPtransfersetup} and was the same as for the NOD donor/NOD KO recipient transfer except for the time points for analysis.
The two donors used in this transfer were 6 week old female NOD-RAG-GFP mice. 
The recipients were 8-10 week old female NOD mice.
\todo{?Chosen ages.}

The following tissues were analysed in the recipient mice:
\begin{itemize}
\item Thymus - The origin of the donor cells. It was analysed in the recipient to see if the donor cells would preferentially migrate to the recipient thymus.
This would give the impression that thymic B cells have specific properties which allow them to traffic back to the thymus.
For example, they may have specific cytokine sensitivity which results in being able to home to the thymus, even following intravenous transfer.
\item Spleen - The spleen was included for analysis as this is the normal site of B cell maturation.
Therefore, it was of interest to see if thymic derived donor cells would move here like conventional B cells do to finish development.
\item Bone marrow - The bone marrow is the normal site of B cell development and therefore was included to see if the developing donor B cells would preferentially migrate here to develop in the same way as conventional B cells
%\item Pancreas - The pancreas contains the Islets of Langerhans which are destroyed in T1D through T and B cell mediated attack.
%It is the site of infiltration therefore it was included in analysis to see if thymic B cells were part of the infiltration process.
%If so, it may indicate that thymic B cells could have a direct role in the pathogenesis of the disease.
\item Pancreatic lymph node - The pancreatic lymph node (PLN) is the draining lymph node of the pancreas.
In T1D, APCs move from the pancreatic islets to the PLN carrying islet antigens to activate T cells and therefore the PLN is in an inflammatory state.
\item Lateral axillary lymph node - This tissue was analysed as a control to compare to the PLN.
Whereas the PLN should be inflamed in T1D, the lateral axillary lymph node should not and can therefore show that any inflammation in the PLN is localised and is not as a result of inflammation throughout the body.
\end{itemize}

The donor thymi were taken from the donor mice and split into CD19\textsuperscript{+} and CD19\textsuperscript{-} fractions using CD19 microbeads (Miltenyi).
Flow cytometric analysis of the separated donor thymi fractions are shown in \cref{subfig:NODdonorseparation} and \cref{subfig:GFPdonorseparation}.
The separations worked well in that the CD19\textsuperscript{-} fractions had very few contaminating B cells (\cref{subfig:WTdonortable}, \cref{subfig:GFPdonortable}).
The CD19\textsuperscript{+} fractions contain almost 100\% of thymic B cells, however, they are contaminated with non B cells.
The aim of the separation was to have a fraction that contained no B cells (CD19\textsuperscript{-} fraction), and a fraction that contained thymic B cells (CD19\textsuperscript{+} fraction) so that any effects of thymic B cells in the recipients could be identified.
With this in mind, the separation worked as there were very few B cells present in the CD19\textsuperscript{-} fraction compared to the CD19\textsuperscript{+} fraction.

%The numbers/phenotypes of donor cells are shown in \cref{subfig:WTdonortable}.
%\new{Unfortunately, it appears that the process of separation into CD19\textsuperscript{+} and CD19\textsuperscript{-} fractions may not have worked very well, shown by the small percentage of B cells in the CD19\textsuperscript{+} fraction.
%It means that in absolute numbers, the number of B cells injected into the CD19\textsuperscript{+} recipient mice was less than double that injected into the CD19\textsuperscript{-} recipient.
%This must therefore be taken into account when comparing the two recipients as they both received fairly small numbers of donor B cells.}

\begin{figure}
	\begin{subfigure}{0.5\textwidth}
	\centering
	\includegraphics[width=\textwidth]{Figures/KOTransferexptsetup.png}
	\caption{Transfer setup}
	\label{fig:KOtransfersetup}
	\end{subfigure}
	\begin{subfigure}{0.5\textwidth}
	\centering
	\includegraphics[width=\textwidth]{Figures/GFPtransfersetup.png}
	\caption{Set up}
	\label{subfig:GFPtransfersetup}
	\end{subfigure}
	\begin{subfigure}{0.5\textwidth}
	\includegraphics[width=\textwidth]{Figures/NODdonor.png}
	\caption{NOD donor cells}
	\label{subfig:NODdonorseparation}
	\end{subfigure}
	\begin{subfigure}{0.5\textwidth}
	\centering
	\includegraphics[width=\textwidth]{Figures/WTdonortable2.png}
	\caption{Donor cell numbers and phenotypes}
	\label{subfig:WTdonortable}
	\end{subfigure}
	\begin{subfigure}{0.5\textwidth}
	\includegraphics[width=\textwidth]{Figures/GFPdonor.png}
	\caption{GFP donor cells}
	\label{subfig:GFPdonorseparation}
	\end{subfigure}
	\begin{subfigure}{0.5\textwidth}
	\centering
	\includegraphics[width=\textwidth]{Figures/GFPdonortable2.png}
	\caption{Donor cell numbers}
	\label{subfig:GFPdonortable}
	\end{subfigure}
\caption[Experimental setup and donor cell populations for transfer experiments]{}
\label{}
\end{figure}
	
	
%\caption[Experimental set up for transfer experiment]{Transfer experiment setup.
%NOD WT thymi taken and split into CD19\textsuperscript{+} and CD19\textsuperscript{-} fractions. 
%Fractions injected into NOD KO recipient mice as shown and recipient mouse thymi, spleens, bone marrow, PLN and axillary LN analysed by flow cytometry at either 4 or 7 day time point.
%Donor cell numbers are shown in table, along with number of transferred B cells.

	
	
%\caption{Transfer experiment setup.
%NOD-RAG-GFP thymi taken and split into CD19\textsuperscript{+} and CD19\textsuperscript{-} fractions. 
%Fractions injected into NOD WT recipient mice as shown and recipient mouse thymi, spleens, bone marrow, PLN and axillary LN analysed by flow cytometry at either 4 or 7 day time point.
%Donor cell numbers are shown in table, along with number of transferred B cells.}
%\label{subfig:Experimentsetup}
%\end{figure}

\subsubsection{NOD to KO}

The aim of this transfer experiment was to see whether thymic B cells could have an effect on the endogenous thymic pro B cell population in the NOD KO recipient.
Further to this, it was of interest to see whether the transferred B cells would be able to survive in the recipient mice.
Previously, transfer of splenic B cells into NOD KO recipients has resulted in B cell death \citep{Serreze1998}, therefore it was wondered whether this would be the case when the transferred B cells were of thymic origin.
The recipient mice were assessed at 7 days and 11 days following transfer of donor cells to determine the initial and long term population of various tissues with mature B cells and/or the initiation of B cell development at the level of pro B cells.


%The experiment was of interest due to the fact a similar thing has been attempted before by \citet{Serreze1998}, whereby NOD B cells were injected into NOD KO recipient mice.
%Interestingly, it was found that the transferred B cells disappeared in the recipient mice between 6-11 days post transfer, hence why I chose 7 and 11 day timepoints for analysis.
%However, the difference was \citet{Serreze1998} injected conventional splenic B cells.
%Here in my transfer experiment, the transferred cells were of thymic origin and since relatively little is known about thymic B cells, it was wondered whether the outcome would be the same. 
%The aim of the experiment was 2 fold.
%One, to see if thymic B cells could survive transfer and two, to see if the presence of thymic B cells had an effect on the frequency of pro B cells in the KO thymus.

The analysis revealed that mature B cells were only present in the recipient mice at the 7 day post transfer time point, see \cref{fig:Transfer}.
Bone marrow, thymus, spleen, pancreas, pancreatic lymph node and axillary lymph nodes were all analysed to look for donor B cell presence, however, B cells are only seen in the the thymus and spleen at the 7 day time point.
In concordance with \citet{Serreze1998}, no B cells were seen in any of the tissues at the 11 day time point.
This could  be that the B cells are being killed off by cytotoxic T cells which have never developed tolerance to B cells due to the B cell absence during T cell development.
On the other hand, due to the fact that the group size is very small (one mouse per group), the results must be viewed with caution.

Although B cells are seen in the thymus and spleen of both recipients at the 7 day time point, it seems that there is a higher frequency of CD19\textsuperscript{+}IgM\textsuperscript{+} cells in the CD19\textsuperscript{+} recipient.
This finding is not surprising as this recipient received more mature B cells in the transferred cells.
However, there are still some B cells seen in the CD19\textsuperscript{-} recipient.
These could be B cells that developed from CD19\textsuperscript{-} progenitors which were injected or they may be contaminating B cells that were transferred.
The experiment was carried out double blind to avoid bias during analysis.

\begin{figure}
	\begin{subfigure}{\textwidth}
	\includegraphics[width=0.5\textwidth]{Figures/CD19posrecipspleen.png}
	\includegraphics[width=0.5\textwidth]{Figures/CD19negrecipspleen.png}
	\caption{Spleen of recipient mice showing frequency of CD19+IgM+ B cells at 7 and 11 days post transfer}
	\end{subfigure}
	\begin{subfigure}{\textwidth}
	\includegraphics[width=0.5\textwidth]{Figures/CD19posrecipthy.png}
	\includegraphics[width=0.5\textwidth]{Figures/CD19negrecipthy.png}
	\caption{Thymus of recipients showing B cell presence/absence at 7 and 11 days post transfer}
	\end{subfigure}
\caption[Donor B cells are found in the spleen of recipient mice at 4 days post transplant]{Figure showing B cell presence/absence in spleen and thymus over time.
Top panel shows B cell presence in the spleen of the CD19+ and CD19- recipients at 7 and 11 days post transfer.
Bottom panel shows the same but in the recipient thymi.
Data shown is gated lymphocytes, single cells, GFP\textsuperscript{+} then analysed on CD19 expression.
n=1 for every group.}
\label{fig:Transfer}
\end{figure}

It was then wondered whether or not the recipient NOD KO mice had an increase in pro B cells in the thymus due to the effect of transferred B cells.
However, it appears that there is no difference in the frequency of pro B cells between the KO control and either recipient (Data not shown).
The results were also very varied and inconsistent therefore the group sizes would need to be increased significantly and the experiment repeated in order to obtain any meaningful results.


\subsubsection{GFP to WT}

The aim of this transfer was to determine whether or not thymic B cells would be drawn back to the thymus in recipient mice following transfer.
To establish this, the thymus, bone marrow, spleen, PLN and axillary LN were analysed by flow cytometry at 4 and 7 days post transfer to see which tissues contained donor GFP\textsuperscript{+}CD19\textsuperscript{+}IgM\textsuperscript{+} B cells.






The results obtained from this experiment suggests that transferred B cells will only migrate to the thymus, spleen and axillary lymph node, as shown in \cref{fig:GFPBcellgraphs}.

Firstly, in the thymus, no GFP\textsuperscript{+} B cells were seen in the CD19\textsuperscript{+} recipient at either time point, shown by the numbers being less than that seen in the NOD WT control, see \cref{subfig:ThyGFPBcells}.
However, the total number of cells in the thymus of each mouse was also different (NOD WT 8.8 x 10\textsuperscript{6}, 4 day CD19+ recipient 4.8 x 10\textsuperscript{6}, 4d CD19- recipient 9.8 x 10\textsuperscript{6}, 7d CD19+ recipient 3.1 x 10\textsuperscript{6}, 7d CD19+ recipient 5.1 x 10\textsuperscript{6}, NOD-RAG-GFP control 7.3 x 10\textsuperscript{6}).
This shows that there are about double the number of thymic cells in the NOD compared to the CD19\textsuperscript{+} recipients at either time point, suggesting that the lower number of GFP+ B cells could be accounted for by the smaller number overall.
This decrease in total thymus cell number may be due to the mouse itself or the preparation of the thymus.
The other recipients have more similar total thymus cell numbers and are therefore more comparable.
Only the CD19\textsuperscript{-} recipient shows an increase in GFP\textsuperscript{+} B cells and even then, only at the 4 day time point.

In the bone marrow, \cref{subfig:BMGFPBcells}, all the recepients look very similar to the NOD control mouse, suggesting that any GFP\textsuperscript{+} B cells seen there are maybe just attributable to noise.

The spleen, showever, does show an increase in GFP\textsuperscript{+} B cells in all recipients at all time points compared to NOD control, \cref{subfig:SpleenGFPBcells}. 
This is interesting as it suggests that transferred GFP\textsuperscript{+} B cells are migrating to the spleen to develop, as would conventional B cells.
The total thymus cell count for the NOD WT is about the same as the 7 day CD19\textsuperscript{-} recipient (~3.7 x 10\textsuperscript{6} cells), which is about half the number of cells in each of the other mice.
However, the number of GFP\textsuperscript{+} B cells seen in all recipient mice is increased compared to the NOD mouse suggesting that the difference cannot be accounted for by the difference in thymus cell number.

In the pancreatic lymph node, \cref{subfig:PLNGFPBcells}, it appears that there is a lot of `noise' in the NOD control as this mouse should not have any GFP and yet the number of GFP\textsuperscript{+} B cells is supposedly far greater than that seen in the NOD-RAG-GFP positive control. 
It seems that there are no GFP\textsuperscript{+} B cells in any recipients, and that this is reasonable due to the very small number seen in the NOD-RAG-GFP positive control mouse.

Finally, in the axillary lymph node \cref{subfig:axLNGFPBcells}, it appears that GFP\textsuperscript{+} B cells are only seen in the CD19\textsuperscript{+} and potentially CD19\textsuperscript{-} recipients at the 4 day time point.
The total cell counts for each recipient and control were all fairly similar therefore it appears that any differences in GFP\textsuperscript{+} B cells cannot be accounted for by differences there.

Overall it appears that GFP\textsuperscript{+} B cells migrate mainly to the spleen, with a some migrating to the thymus and axillary lymph node.
This suggests that thymic derived B cells do not on the whole migrate back to the thymus following transfer.
However, in the thymus, the CD19\textsuperscript{-} recipient is the only one to show an increase in GFP\textsuperscript{+} B cells. 
This could be due to chance or contaminating B cells, or it may be due to the fact that this fraction contains thymic-derived CD19\textsuperscript{-} B cell progenitors which may need to be in the thymus to develop into B cells.
This could indicate that thymic B cells need to develop initially in the thymus, but following this, they can continue development in the spleen, similar to conventional B cells.
In order to investigate this further, the group size would need to be significantly increased to see whether this difference is due to chance or is a real observation.

\begin{figure}
	\begin{subfigure}{0.45\textwidth}
	\includegraphics[width=\textwidth]{Figures/ThyGFPBcells.pdf}
	\caption{Thymus}
	\label{subfig:ThyGFPBcells}
	\end{subfigure}
	\begin{subfigure}{0.45\textwidth}
	\includegraphics[width=\textwidth]{Figures/BMGFPBcells.pdf}
	\caption{Bone marrow}
	\label{subfig:BMGFPBcells}
	\end{subfigure}
	\begin{subfigure}{0.45\textwidth}
	\includegraphics[width=\textwidth]{Figures/SpleenGFPBcells.pdf}
	\caption{Spleen}
	\label{subfig:SpleenGFPBcells}
	\end{subfigure}
	\begin{subfigure}{0.45\textwidth}
	\includegraphics[width=\textwidth]{Figures/PLNGFPBcells.pdf}
	\caption{Pancreatic lymph node}
	\label{subfig:PLNGFPBcells}
	\end{subfigure}
	\begin{subfigure}{\textwidth}
	\centering
	\includegraphics[width=0.45\textwidth]{Figures/axLNGFPBcells.pdf}
	\caption{Axillary lymph node}
	\label{subfig:axLNGFPBcells}
	\end{subfigure}
\caption[Transferred B cells migrate preferentially to the spleen and thymus]{Graphs showing where transferred B cells migrate to.
Each graph shows the number of GFP\textsuperscript{+} B cells (CD19\textsuperscript{+}IgM\textsuperscript{+}) present in each tissue in each receipient/control. NOD WT control was included as a negative control as it has no GFP therefore any GFP\textsuperscript{+} B cells can be classed as `noise'. 
NOD-RAG-GFP was included as a positive control. 
Recipient mice were analysed at 4 and 7 days post transfer. n=1 for each group.}
\label{fig:GFPBcellgraphs}
\end{figure}

\subsection{The NOD thymus can support BcR rearrangement}

Initial rearrangement of the BcR takes place at the pro to pre B cell transition phase, and the finding that pre B cells are present in the NOD thymus suggests that the thymus can support early rearrangments of the BcR.
To confirm this, NOD-RAG-GFP reporter mice were utilised.
These mice express GFP under the control of the RAG promoter, therefore, during receptor rearrangement when RAG is being transcribed, GFP is also produced and can act as a marker for RAG activity.

Further to this, while transcription of RAG and GFP cease simultaneously, GFP degrades slowly over time with a half life of approximately 56 hours in vivo \citep{McCaughtry2007}.
This means that the intensity of GFP signal seen on flow cytometry correlates with how recently RAG/GFP transcription was active.
For example, as shown in \cref{subfig:BMRAG} the GFP\textsuperscript{high} population in the bone marrow represents the developing B cells which are actively rearranging their receptor.
The GFPG\textsuperscript{low} population shows the cells that have recently rearranged their receptor and subsequently deactivated RAG/GFP transcription, therefore, the GFP signal is decreased.
The GFP\textsuperscript{-} population represents cells which rearranged their receptor more than 56 hours previously.

Murine strains of RAG-GFP reporter mice have been invaluable at documenting the differing frequencies of developing, newly developed and long term resident thymic T regulatory cells depending on co-stimulatory signals \toref{Steve's paper}.
Thus, NOD-RAG-GFP mice enabled analysis not only of BcR rearrangment in developing B cell populations, but also the proportion of developing, newly developed and long term resident thymic B cells.

To investigate whether the thymus may be able to support BcR rearrangement, cells of the B cell lineage were identified by expression of CD19 then these cells were interrogated for GFP expression.
As shown in \cref{subfig:RAGhighlownegthyB}, there are CD19\textsuperscript{+}GFP\textsuperscript{high} cells within the thymus of NOD mice at 4, 7 and 11 weeks of age.
This finding suggests that these B cells are actively rearranging their receptor in the thymus.
If these cells had migrated to the thymus from the bone marrow, it would be expected that the GFP expression would be lower due to GFP decay during the migration process.
Further to this, as shown in \cref{subfig:RAGhighlowneggraph}, it appears that the frequency of CD19\textsuperscript{+}GFP\textsuperscript{+} cells decreases as mice age.
In particular, the frequency of CD19\textsuperscript{+}GFP\textsuperscript{low} cells decreases significantly (\todo{p value}) between 4 and 11 weeks of age.
Conversely, the frequency of the CD19\textsuperscript{+}GFP\textsuperscript{-} increases significantly between 4 and 7 weeks of age.
The significant decrease in CD19\textsuperscript{+}GFP\textsuperscript{low} cells in the NOD thymus could show that there is a decreased frequency of newly developed B cells in the thymus of older NOD mice.
This could be due to a decrease in B cell development, or increased emigration of newly developed B cells.
This is an interesting result as it could suggest that the support of early BcR rearrangement in the thymus is limited to younger mice.



\begin{figure}
	\begin{subfigure}{\textwidth}
	\includegraphics[width=\textwidth]{Figures/RAGhighlownegthyB.png}
	\caption{}
	\label{subfig:RAGhighlownegthyB}
	\end{subfigure}
	\begin{subfigure}{0.5\textwidth}
	\centering
	\includegraphics[width=0.7\textwidth]{Figures/7wkBMRAG.png}
	\caption{}
	\label{subfig:BMRAG}
	\end{subfigure}
	\begin{subfigure}{0.5\textwidth}
	\centering
	\includegraphics[width=0.7\textwidth]{Figures/7wktotalthyRAG.png}
	\caption{}
	\label{subfig:totalthyRAG}
	\end{subfigure}
	\begin{subfigure}{\textwidth}
	\includegraphics[width=\textwidth]{Figures/RAGhighlownegative.pdf}
	\caption{}
	\label{subfig:RAGhighlowneggraph}
	\end{subfigure}
\caption[Some CD19\textsuperscript{+} cells in the thymus appear to be expressing RAG]{CD19\textsuperscript{+} cells in the NOD thymus express RAG. 
(a) shows the RAG expression of CD19+ cells in the thymus of NOD-RAG-GFP reporter mice at 4, 7 and 11 weeks of age. Each plot is representative of 4/5 sex-matched mice of given age. Single cell suspensions prepared then stained for flow cytometry. Samples gated on acquisition for CD19\textsuperscript{+} cells. For analysis, samples gated lymphocytes, single cells, CD19\textsuperscript{+}. (b) shows RAG expression in the NOD bone marrow, gated lymphocytes, single cells, CD19\textsuperscript{+}. Data representative several mice. (c) shows the total thymic RAG expression used to set gates of RAG\textsuperscript{high}, RAG\textsuperscript{low} and RAG-. Data gated on lymphocytes and single cells. (d) is a graph showing the change in frequency of RAG\textsuperscript{high}, RAG\textsuperscript{low} and RAG\textsuperscript{-} cells with mouse age. Significance determined using one-way ANOVA (p<0.0001) and Tukey's test.} 
\end{figure}

\todo{Correct axis labels to GFP}


\subsection{Early B cell progenitors are present in the NOD thymus}
\label{subsec:earlyprogens}

\subsubsection{Introduction to progenitor work}

As shown in \cref{subfig:IncthyBcells}, there is an age-related, significant increase in thymic B cells in NOD mice compared to B6 control mice that do not develop T1D.
The age-related increase correlates with the onset of insulitis in NOD mice and increased release of autoreactive T cells.
Given this difference between the NOD and B6 mouse, and the evidence that B cell development is supported in the thymus, the question arose as to whether thymic seeding progenitor cells with B cell potential are present within the NOD thymus at higher frequencies compared to control B6 mice.

For this, it was important to first of all decide on the specific progenitor that would be looked for.
Although the B cell commitment pathway remains undefined, as mentioned in \cref{subsec:Bcelldevelopment}, it appears that a so-called BLP may be the best described B cell progenitor.
In which case, the cells of interest would be Sca-1\textsuperscript{low} c-kit\textsuperscript{low} Flt3\textsuperscript{+} IL-7Ra\textsuperscript{+} Ly6D\textsuperscript{+} \citep{Mansson2010, Inlay2009, Zhang2013}.

However, the matter was complicated further due to the fact research into B cell commitment and development is focussed normally on normal development within the bone marrow.
Research into the development of B cells in the thymus, therefore, requires the assumption that the developmental pattern will match that of the bone marrow.
While it is not known how well the developmental pattern in the thymus mirrors that of the bone marrow, it is a good place to start due to the wealth of literature relating the B cell development in the bone marrow.

The aim of the analysis was to determine whether or not the frequencies of BLPs in the NOD mouse were the same as that seen in the non diabetic B6 control and in the NOD KO.
This would give insight into two areas.
Firstly, if NOD mice have an increased frequency of B cell progenitors in their thymus compared to B6 mice, this could potentially explain the increased number of B cells in the NOD thymus.
And secondly the analysis could ask if BLPs are affected in the same way as pro B cells by the presence or absence of a mature B cell.
This would be shown by a decreased frequency of BLPs in the NOD KO compared to the NOD.

Mice used were 6-8 weeks old.
This age was chosen as this is the point that B cells start to significantly increase compared to B6 controls.
Therefore, it is a good point to investigate the development of the B cells as any differences between the two strains should be most apparent.



\subsection{Magnetic-assisted cell sorting optimisation}

Due to the small size of progenitor populations in comparison to mature cell populations in the thymus, it was first necessary to deplete the thymus of the majority of mature cells before looking for progenitors.
For depletion, magnetic-activated cell sorting (MACS) was used.
There are many different kits available and the decision on which to use for the experiments was taken after investigating the efficiency and yield from Miltenyi lineage depletion kits and Qiagen BioMag goat anti-rat IgG beads (see \cref{Methods:MACSdepletion}).

Firstly, Miltenyi columns and lineage depletion beads were tested to assess their efficiency.
To start with, only one round of depletion was carried out which gave an efficiency of \todo{look at efficiencies}, see \cref{subfig:1rounddep}.
This efficiency was improved to \todo{add efficiency of 2 rounds} when two rounds of depletion were carried out \cref{subfig:2rounddep}. 
However, while this method gave a very pure sample with very few mature cells remaining, the yield was not very good.

\begin{figure}
\includegraphics[width=\textwidth]{Figures/Qiagenbeads.png}
\caption{Flow cytometric analysis of lineage depletion using Qiagen beads}
\label{fig:Qiagenbeads}
\end{figure}

%\begin{figure}
	%\begin{subfigure}{\textwidth}
	%\includegraphics[width=0.8\textwidth]{Figures/1rounddepletion.png}
	%\caption{One round of depletion using Miltenyi lineage depletion beads and columns}
	%\label{subfig:1rounddep}
	%\end{subfigure}
	
	%\begin{subfigure}{0.8\textwidth}
	%\includegraphics[width=\textwidth]{Figures/2rounddepletion.png}
	%\caption{Two rounds of depletion using Miltenyi lineage depletion beads and columns}
	%\label{subfig:2rounddep}
	%\end{subfigure}
	
	%\begin{subfigure}{0.8\textwidth}
	%\includegraphics[width=\textwidth] {Figures/Qiagenbeads.png}
	%\caption{Qiagen Beads}
	%\label{subfig:Qiagen}
	%\end{subfigure}
	
%\caption[Optimisation of MACS lineage depletion]{Figure showing the different methods of depletion. 
%Methods for each outlined in \cref{Methods:MACSdepletion}.
%FACS plots gated lymphocytes, single cell CD4\textsuperscript{-}CD8\textsuperscript{-}, then for Miltenyi methods CD19\textsuperscript{-}CD11c\textsuperscript{-} and for Qiagen beads B220\textsuperscript{-}.}
%\end{figure}

Qiagen beads were also tested as an alternative to the Miltenyi kits. 
As shown in \cref{subfig:Qiagen}, the efficiency of this method was also very good and the yield was also much improved compared to the Miltenyi beads therefore this was the chosen method of depletion.


\subsection{BLPs are present in NOD mouse}

In order to assess the frequency of BLPs present in the thymus of NOD, NOD KO and B6 mice, it was first necessary to establish a suitable gating strategy.
Firstly, data was gated to find the CLP population within the thymus.
For this a lymphocyte and single cell gate was applied then data was gated to anaylse only cells which were Sca-1\textsuperscript{low}c-kit\textsuperscript{low}Flt3\textsuperscript{+}IL-7R$\alpha$\textsuperscript{+}.
From this population, it is then possible to determine which of these cells are BLPs and therefore likely to be B cell committed, depending on their Ly6D expression.
This gating pattern is shown in \cref{subfig:BLPgating}.

When examining the frequencies of BLPs within the thymic CLP population, it appeared that BLPs are present in the thymus of all strains of mice.
However, interestingly, the frequency of BLPs in the thymic CLP population was significantly decreased in the NOD and NOD KO mice compared to the B6 controls, as shown in \cref{subfig:BLPgraph}.
This result was unexpected due to the increased thymic B cell population seen in NOD thymi compared to B6 thymi at this age.

The presence of BLPs in the B6 thymus could suggest that they are the normal progenitor for the normal population of thymic B cells seen in nondiabetic animals.
However, the decreased population of BLPs seen in the NOD mouse could suggest that there may potentially be an alternative mechanism by which B cells increase in the thymus of NOD mice.
With this is mind, other potential B cell development patterns must be explored.
%Interestingly, the analysis gave very inconsistent results for both the WT and KO NOD mice, as shown in \cref{subfig:BLPgraph}.
%The frequencies of each population of cells varied widely over the 3 NOD WT and 4 NOD KO thymi. 
%For example, while the frequencies of Sca-1\textsuperscript{low} and c-kit\textsuperscript{low} cells remained fairly similar for both NOD KO and NOD mice, beyond this point, the percentages of IL-7R$\alpha$\textsuperscript{+}, Flt3\textsuperscript{+} and Ly6D\textsuperscript{+} showed a large amount of variation.
%However, the same was not seen in the B6 control mice.
%These mice showed very consistent frequencies of each population.

%Whilst the data is not hugely useful in determining difference in presence of B cell progenitors, it does suggest that in the B6 mouse, the pattern of development is very well controlled and therefore consistent between mice.
%This contrasts with the mice of NOD background which show huge variablity suggesting that the pattern of development in these mice is much less well controlled.
%This potential lack of control could be contributing to the increase in B cells in the thymus.

%However, despite the variable data from the NOD mice, the data may suggest that actually there are more BLPs present in the B6 mouse compared to the NOD mouse.
%This is surprising due to the significantly increased population of B cells seen in the NOD thymus.


\begin{figure}
	\begin{subfigure}{\textwidth}
	\includegraphics[width=\textwidth]{Figures/BLPgating.png}
	\caption{Gating pattern used to identify BLPs}
	\label{subfig:BLPgating}
	\end{subfigure}
	\begin{subfigure}{\textwidth}
	\centering
	\includegraphics[width=0.6\textwidth]{Figures/Ly6D.pdf}
	\caption{}
	\label{subfig:BLPgraph}
	\end{subfigure}
\caption[BLPs are significantly decreased in the NOD thymus compared to the B6 thymus]{BLPs are present in the NOD thymus. 
Top panel shows gating pattern to look for BLPs in the thymus of mice. Lymphocyte and single cell gate not shown.
Bottom panel shows the frequency of each marker within the population of the previous marker.
One way ANOVA (p=0.0003) and post hoc Tukey's test carried out on Ly6D\textsuperscript{+} population revealing a significantly increased population of BLPs in the B6 mice compared to both NOD and NOD KO thymi.
11 mice were used, all aged 6-8 weeks. 
The experiment was carried out blind to avoid bias. 
NOD WT n=3, NOD KO n=3, B6 n= 4.}
\label{fig:BLPs}
\end{figure}

Further to this the presence of BLPs in the NOD and NOD KO thymi were compared in older mice, aged 12-13 weeks which correlates with the onset of CTL development and $\beta$ cell destruction \cref{fig:diseasecourse}.
Unfortunately there were no control B6 mice available at the time, however, it is still possible to suggest whether mature B cells have an effect on the thymic BLP population.
Mice of this age have a significantly increased thymic B cell population compared to nondiabetic B6 controls \cref{subfig:IncthyBcells}.

As shown in \cref{fig:olderBLPs}, there is no difference in the frequency of Ly6D\textsuperscript{+} BLPs within the CLP population.
This may suggest that the action of a mature B cell to increase the pro B cell population in B cell suffcient mice as opposed to B cell deficient mice, is after the BLP stage.

\begin{figure}
\centering
\includegraphics[width=0.6\textwidth]{Figures/NODvKOBLPs.pdf}
\caption[The frequency of thymic BLPs is the same in both NOD and NOD KO thymi]{Ly6D\textsuperscript{+} BLP presence within the thymic CLP population was assessed by flow cytometry and found to not be statistically significantly different between the NOD and NOD KO thymus. Single cell suspensions were prepared then depleted of CD4, CD8 and CD19 mature cells using Qiagen BioMag goat anti-rat beads. Cells were then incubated with fluorescently-labelled analytic antibodies and assessed by flow cytometry.
Statistical significance was determined by the Mann Whitney test and the two strains were found to be not statistically significant. NOD n=5, NOD KO n=5.}
\label{fig:olderBLPs}
\end{figure}


\subsection{The NOD thymus contains B cell development transcription factors}
\label{subsec:TFs}

If B cells are developing within the thymus, this begs the question as to whether the NOD thymus is providing an environment more conducive to B cell development compared to nondiabetic controls. 
To investigate this, primers were designed that were specific for B cell development factors such as E2A, EBF, Pax5, VPreB and CXCL12.
%\todo{look in Hagman2006, Welinder2011, Mansson2008, Cobaleda2007, Radtke2013, Riley2013, Tokoyoda2004 for TF info}.

To date, only non-quantitative PCR has been carried out to compare whether or not B cell development transcription factors are present in the NOD bone marrow and thymus.
Since normal B cell development occurs in the bone marrow, these transcription factors would be expected there and indeed they were seen there, as shown in \cref{fig:gels}. 
Interestingly, all the genes, besides CXCL12, were also present in the NOD thymus too.
This indicates that the thymic environment may also be allowing B cell development.
The lack of CXCL12 in the thymus is not a surprising result as it is a cytokine responsible for holding developing B cells in the bone marrow niche and therefore, may be more bone marrow specific rather than B cell specific.
\new{This raises the question as to what chemokine is holding developing B cells in the thymic niche to enable completion of their development if it isn't CXCL12.}


\begin{figure}
	\begin{subfigure}{0.5\textwidth}
	\centering
	\includegraphics[width=\textwidth]{Figures/E2A.pdf}
	\caption{E2A}
	\end{subfigure}
	\begin{subfigure}{0.5\textwidth}
	\centering
	\includegraphics[width=\textwidth]{Figures/EBF.pdf}
	\caption{EBF}
	\end{subfigure}
	\begin{subfigure}{0.5\textwidth}
	\centering
	\includegraphics[width=\textwidth]{Figures/sPax5.pdf}
	\caption{Pax5}
	\end{subfigure}
	\begin{subfigure}{0.5\textwidth}
	\centering
	\includegraphics[width=\textwidth]{Figures/VPreB.pdf}
	\caption{VPreB}
	\end{subfigure}
	\begin{subfigure}{\textwidth}
	\centering
	\includegraphics[width=0.5\textwidth]{Figures/CXCL12.pdf}
	\caption{CXCL12}
	\end{subfigure}
\caption[Non quantitative PCR reveals B cell development transcription factors are present in the NOD thymus]{B cell development transcription factors are present in NOD thymi.
Gel electrophoresis of non-quantitative PCR products shows presence of B cell development transcription factors.
Lanes from left to right; Thymus, Bone marrow, 100 bp marker. 
600bp mark showed in red box.
Expected product size is shown in blue box.
Gels were run in 3\% agarose in 1 x TAE buffer. Gels supplemented with 1 $\mu$L ethidium bromide per 25 $\mu$L of 1 x TAE buffer. Gels run at constant 70V. }
\label{fig:gels}
\end{figure}



%\subsection{Conclusion}

%\new{Data to date has so far given evidence that B cells could develop intrathymically.
%I have shown that cells with the phenotype of pro and pre B cells are present in the thymus, suggesting it may be an environment that can support these developing B cells.
%I have also shown that some CD19\textsuperscript{+} cells in the thymus of NOD-RAG-GFP mice are expressing high levels of GFP, suggesting active BcR rearrangement and development.
%It also appears that the thymus is capable of B cell development transcription factor expression.
%This could suggest that the thymus is able to provide an environment conducive to B cell development.
%However, it is necessary to look at B cell development transcription factors quantitatively and then compare this expression to that seen in the B6 thymus.

%There is also evidence to suggest that BLPs are present in the thymus of NOD and B6 mice, though interestingly, it appears that B6 mice may have an increased frequency of BLPs in their thymus compared to the NOD mouse.
%This is not what would be expected due to the significantly increased population of B cells in the NOD thymus.}

\section{Some RAG+ cells in the thymus express B and T cell markers}

\new{B cells normally develop from progenitor BLPs in the bone marrow.
Therefore, initially, BLPs were looked for in the NOD thymus to assess whether or not the normal B cell developmental pattern could be seen in the thymus.
However, following the results suggesting that NOD thymi may have less thymic BLPs compared to the B6 controls (\cref{subfig:BLPgraph}), an increase in early B cell progenitors is unlikely to account for the increase in thymic B cells in the NOD.
This would suggest that there is an alternative pathway of B cell development in the thymus that is different to the pathway seen in the bone marrow.
With this in mind, it was noted that a large proportion (>60\%) of GFP\textsuperscript{+}CD19\textsuperscript{+} cells in the NOD thymus, were also CD4\textsuperscript{+}CD8\textsuperscript{+}.
This was of interest and raised the question as to whether a cell with dual B and T cell markers exist in the NOD thymus and may give rise to bone-fide mature B cells.
It may be that T cells are able to transform into B cells and that these cells are the midpoint of transition.}

\subsection{A large proportion of RAG\textsuperscript{+}CD19\textsuperscript{+} cells in the NOD thymus express CD4 and CD8}

When GFP\textsuperscript{+}CD19\textsuperscript{+} cells in the NOD thymus were interrogated for CD4 and CD8 expression, as shown in \cref{subfig:ThyRAGCD19DP}, a large percentage of these cells were CD4\textsuperscript{+}CD8\textsuperscript{+}.
This was an interesting finding as it suggests there may be a point where cells are not yet fully lineage committed and are expressing markers of both T (CD4, CD8) and B (CD19) cells.
These cells all have a high GFP signal and therefore are actively transcribing the protein, indicating that RAG is also active.
This suggests that these cells are expressing markers of both T and B cells as well as rearranging a receptor, though it is not possible to tell which one.

Unfortunately, due to lack of control mice of the correct age, it is not possible to say whether this is a normal pysiological process seen in nondiabetic control mice as well as NOD mice or whether it is unique to the NOD.
Repetitions of the experiments would need to be carried out with the inclusion of B6 control mice.
However, due to normal B cell development occurring in the bone marrow, the thymus and bone marrow were compared for presence of RAG\textsuperscript{+}CD19\textsuperscript{+}CD4\textsuperscript{+}CD8\textsuperscript{+} cells and the bone marrow served as a tissue control.

Mice of 4 and 7 weeks of age were analysed and a representative 4 week old NOD thymus and bone marrow are shown in \cref{fig:RAGCD19DP}. 
Here, the difference between the thymus \cref{subfig:ThyRAGCD19DP} and bone marrow \cref{subfig:BMRAGCD19DP} is shown.
While the thymus contains a significant population of RAG\textsuperscript{high}CD19\textsuperscript{+}CD4\textsuperscript{+}CD8\textsuperscript{+} cells, the bone marrow contains none.

The lack of RAG\textsuperscript{high}CD19\textsuperscript{+}CD4\textsuperscript{+}CD8\textsuperscript{+} cells in the bone marrow suggests it is unlikely that these cells are a normal part of B cell development due to the bone marrow being the normal B cell development site.
It is also unlikely that they are a normal part of T cell development as T cell commitment is thought to occur fully in the DN development stage, therefore when developing T cells are still CD4\textsuperscript{-}CD8\textsuperscript{-}.

\new{Further to this, if these results were an artefact, it would be likely that they would be seen in the bone marrow due to the normal trafficking of T cells to the bone marrow.}

This suggests that they are thymus specific and it may highlight another abnormality within the NOD thymus that may or may not be related to the increased population of thymic B cells.
It could potentially suggest an alternative mechanism by which B cells are increased in the NOD thymus, and that is that a cell is beginning to commit to the T cell lineage, then at some point receiving a signal which is causing it to begin to switch lineages and start expressing B cell markers.
This potential redifferentiation could explain the existence of both B and T cell markers on the cells. 
However, further research would be required to explore this further.

These cells are present in NOD mice at both 4 and 7 weeks of age and it was therefore investigated whether or not there was a significant difference in the percentages of these cells between the two age groups. 
The bone marrow is also included in the figure as a comparison.
However, the results show that there is no difference in presence of RAG\textsuperscript{high}CD19\textsuperscript{+}CD4\textsuperscript{+}CD8\textsuperscript{+} cells between the two ages.


\begin{figure}	
	\begin{subfigure}{\textwidth}
	\includegraphics[width=\textwidth]{Figures/Thymus1RAGCD19DP.png}
	\caption{Thymus}
	\label{subfig:ThyRAGCD19DP}
	\end{subfigure}
	\begin{subfigure}{\textwidth}
	\includegraphics[width=\textwidth]{Figures/BM1RAGCD19DP.png}
	\caption{Bone marrow}
	\label{subfig:BMRAGCD19DP}
	\end{subfigure}
	\begin{subfigure}{\textwidth}
	\centering
	\includegraphics[width=0.8\textwidth]{Figures/RAGCD19CD4CD8ThyBM.pdf}
	\caption{}
	\label{BMvThyDPgraph}
	\end{subfigure}
\caption[There are GFP\textsuperscript{+}CD19\textsuperscript{+}CD4\textsuperscript{+}CD8\textsuperscript{+} cells present in the NOD thymus] {RAG\textsuperscript{+}CD19\textsuperscript{+}CD4\textsuperscript{+}CD8\textsuperscript{+} cells are present in the NOD thymus but not bone marrow.
Top panel shows representative FACS plot of presence of these cells in the thymus. A CD19 gate was used on acquisition of data so that 100\% of CD19\textsuperscript{+} cells were acquired with only ~2\% of CD19\textsuperscript{-} cells.
Middle panel shows absence of the cells in the bone marrow.
Bottom panel shows no significant difference in frequency of these cells between 4 and 7 weeks of age in the thymus.
Also shows the lack of GFP\textsuperscript{+}CD19\textsuperscript{+}CD4\textsuperscript{+}CD8\textsuperscript{+} cells in the bone marrow.
4 week old NOD-RAG-GFP n=5, 7 week old NOD-RAG-GFP n=4.}
\label{fig:RAGCD19DP}
\end{figure}

\todo{I've got a bit lost here}

Following analysis of the presence of CD19\textsuperscript{+}CD4\textsuperscript{+}CD8\textsuperscript{+} cells on both GFP\textsuperscript{high} and GFP\textsuperscript{low} cells, the expression of IgM on these cells was assessed.
Interestingly, as shown in \cref{subfig:IgMallpos}, the IgM expression on these cells differed depending on the GFP status.
CD19\textsuperscript{+}CD4\textsuperscript{+}CD8\textsuperscript{+} cells that were GFP\textsuperscript{high} appeared to have much higher IgM expression compared to those that are GFP\textsuperscript{-}.

\begin{figure}
	\begin{subfigure}{\textwidth}
	\includegraphics[width=\textwidth]{Figures/IgMallpos.png}
	\caption{}
	\label{subfig:IgMallpos}
	\end{subfigure}
	\begin{subfigure}{\textwidth}
	\centering
	\includegraphics[width=0.7\textwidth]{Figures/IgMhighGFP.pdf}
	\caption{}
	\label{subfig:IgMallposgraph}
	\end{subfigure}
\caption[IgM expression is higher on CD19\textsuperscript{+}CD4\textsuperscript{+}CD8\textsuperscript{+} cells that are GFP\textsuperscript{high} compared to those that are GFP\textsuperscript{-} in the NOD thymus]{IgM expression is higher on CD19\textsuperscript{+}CD4\textsuperscript{+}CD8\textsuperscript{+} cells that are GFP\textsuperscript{high} compared to those that are GFP\textsuperscript{-} in the NOD thymus.
Single cell suspensions were prepared and then incubated with analytic flourescently-labelled antibodies for flow cytometric analysis.
Data was acquired using a CD19 gate whereby 100\% of CD19\textsuperscript{+} cells were acquired along with only approx 2\% of all other cells. For analysis, the data was gated lymphocytes, single cells, GFP\textsuperscript{high} or GFP\textsuperscript{-}, CD19\textsuperscript{+}, CD4\textsuperscript{+}CD8\textsuperscript{+} then analysed according to IgM expression.
The top panel shows representative plots of IgM expression on CD19\textsuperscript{+}CD4\textsuperscript{+}CD8\textsuperscript{+} that are either GFP\textsuperscript{high} or GFP\textsuperscript{-}.
The bottom panel shows the frequencies of IgM\textsuperscript{high} cells within either the GFP\textsuperscript{high} or GFP\textsuperscript{-} populations of CD19\textsuperscript{+}CD4\textsuperscript{+}CD8\textsuperscript{+} cells.
Comparisons of frequencies of IgM\textsuperscript{high} cells between GFP\textsuperscript{high} and GFP\textsuperscript{-} populations at each age were analysed statistically using the Mann Whitney test and found to be statistically significantly different at 7 weeks (p=0.0286) but not 4 weeks of age.
Gates were set using bone marrow and spleen samples to set levels of IgM.
n=4/5 NOD-RAG-GFP mice.}
\label{fig:IgMallpos}
\end{figure}


\subsection{RAG\textsuperscript{+}IgM\textsuperscript{+}TcR$\beta$\textsuperscript{+} cell presence in the NOD thymus}

\new{While GFP\textsuperscript{+}CD19\textsuperscript{+}CD4\textsuperscript{+}CD8\textsuperscript{+} could potentially be a lymphocyte transitioning from one lineage to another, this is not the only hypothesis.
It may be that a developing T or B cell starts to receive signals which cause it to express markers of both the T and B cell lineage.
If this was the case it was therefore wondered if this cell would continue development and eventually express the receptor for both lineages.
In an attempt to rule out this possibility, IgM\textsuperscript{+}TcR$\beta$\textsuperscript{+} cells were looked for in the NOD thymus.

%Following the finding of RAG\textsuperscript{+}CD19\textsuperscript{+}CD4\textsuperscript{+}CD8\textsuperscript{+} cells in the thymus, it was then of interest to see if these cells with characteristics of both T and B cells progressed to expressing both a TcR and BcR.
%This would suggest that these cells were somehow continuing further down the developmental pathway and that the Rag was activated and rearranging both a T and B cell receptor.

%To investigate this, the RAG\textsuperscript{+}CD19\textsuperscript{+} were interrogated to see if any expressed both IgM and TcR$\beta$.
As shown in \cref{fig:RAGIgMTcRpos}, it appears that there is a very small population of RAG\textsuperscript{+}IgM\textsuperscript{+}TcR$\beta$\textsuperscript{+} in the thymi of NOD  mice.
None can be seen in the NOD KO which is unsurprising as they are unable to make IgM.
Again, the bone marrow was included as a comparative tissue control.
Data for the bone marrow is not shown as none of the samples showed any RAG\textsuperscript{+}IgM\textsuperscript{+}TcR$\beta$\textsuperscript{+} cells at all (Frequencies = 0\%).

This gives the impression that the RAG\textsuperscript{+}CD19\textsuperscript{+}CD4\textsuperscript{+}CD8\textsuperscript{+} cells do not progress to displaying both a T and B cell receptor.
This could suggest that these cells are just late at committing to a lineage, or it may be that they are the mid point of either a T or B cell transitioning to the other.
More extensive investigation into this area is required.}


\begin{figure}
	\begin{subfigure}{\textwidth}
	\includegraphics[width=\textwidth]{Figures/NODKOIgMTcR.png}
	\caption{Representative NOD WT and NOD KO thymi looking for RAG+IgM+TcR+ cells. CD19 on acquisition, then lymphocytes, single cells, RAG+ then analysed on IgM and TcR}
	\label{subfig:BMvThyRAGIgMTcR}
	\end{subfigure}
	\begin{subfigure}{\textwidth}
	\centering
	\includegraphics[width=0.8\textwidth]{Figures/IgMTcR.pdf}
	\caption{}
	\label{subfig:IgMTcRposgraph}
	\end{subfigure}
\caption[There is a very small population of IgM\textsuperscript{+}TcR$\beta$\textsuperscript{+} cells in the NOD thymus]{There is a small population of GFP\textsuperscript{+}IgM\textsuperscript{+}TcR$\beta$\textsuperscript{+} cells in the NOD WT thymus. 
Top panel shows a representative NOD WT thymus and NOD KO thymus.
Gated lymphocytes, single cells, GFP\textsuperscript{+} then analysed on IgM and TcR$\beta$ expression.
GFP\textsuperscript{+}IgM\textsuperscript{+}TcR$\beta$\textsuperscript{+} cells are only present in NOD WT and not NOD KO.
Bottom panel shows frequency of GFP\textsuperscript{+}IgM\textsuperscript{+}TcR$\beta$\textsuperscript{+} cells is significantly lower between each strain of mouse. NOD KO < NOD WT < FVB. FVB mice were included as a control.
Results analysed using one-way ANOVA (p=0.0002) and post hoc Tukey's test which revealed a statistically significant differences between the NOD thymus and all other groups.
All mice were RAG-GFP reporter mice and 11 weeks old. NOD WT n=3, NOD KO n=5, FVB n=4.
Thymi and bone marrow from all 12 mice were analysed blind so as not to bias results.}
\label{fig:RAGIgMTcRpos}
\end{figure}



%TcRpos IgMpos for figs - 24.11.14 


%\subsection{IgM\textsuperscript{+}TcRb\textsuperscript{+} cell presence}

%The search for IgM\textsuperscript{+}TcR$\bet$\textsuperscript{+} cells was then widened to exclude Rag\textsuperscript{+} cells and to include thymic and bone marrow tissue from NOD, KO and B6 mice.




%Whether or not there are mature cells present in other tissues in the body that are expressing TcR and IgM was then explored.
%To do this, flow cytometry was carried out on thymic samples from NOD, NOD KO and B6 mice using antibodies specific to TcRb and IgM.
%This time, the spleen was also investigated alongside the thymus and bone marrow to provide another comparison and to widen the search for IgM+TcR+ cells irrelevant of RAG expression.
%Mice of various different ages were investigated and the results are shown in \fig{figure showing TcR+IgM+ cells (or not!)} .
%Isotype controls were also carried out to help with determining whether cells were expressing TcR or IgM or not.
%The data from the isotype controls in shown in \fig{isotype controls}.

%\fig{Spleen, BM, Thymus of IgM+TcR+ cells. NOD v B6 (?Frozen cells)}


%\subsection{Conclusion}

%\new{Whilst there appears to be significant populations of RAG\textsuperscript{high}CD19\textsuperscript{+}CD4\textsuperscript{+}CD8\textsuperscript{+} cells in the NOD thymus, it appears that these cells, on the whole, do not progress to displaying both a TcR and BcR.
%It may, therefore, be that RAG\textsuperscript{high}CD19\textsuperscript{+}CD4\textsuperscript{+}CD8\textsuperscript{+} cells are just B or T cells which are late to commit to a lineage, or it may be that they are the midpoint of a T cell becoming a B cell or vice versa.
%Further investigation, for example culturing of RAG\textsuperscript{high}CD19\textsuperscript{+}CD4\textsuperscript{+}CD8\textsuperscript{+} cells to determine their fate, would be beneficial.}

%\subsection{Conclusion}

%Following the data obtained above, it appears that B cells derived from the thymus don't have a strong preference for migrating back to the thymus following transfer and instead, most migrate to the spleen.
%This could suggest that the function of a thymic B cell is in the spleen, or that thymic B cells do not have the ability to return to the thymus either once they have been removed from the thymus, or following transfer (for example, the necessary features required to home to the thymus may be lost during transfer).

