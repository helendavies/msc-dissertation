% !TEX root = ../Dissertation.tex
%===================================================================================================

\chapter{Chapter 4 Title}

\Cref{fig:matlabGsrMetric} is an example of including code. It's not very nice, it was added quite last-minute. But it works.

\begin{figure}[h]
	\centering
	
\begin{minipage}{0.8\textwidth}\smaller\begin{verbatim}
%%%%%%%%%%%%%%%%%%
%% ArousalMetric.m
%%%%%%%%%%%%%%%%%%

% Find peaks and troughs in raw signal
[pks, pLocs] = findpeaks(filtered_signal);
[trs, tLocs] = findpeaks(-filtered_signal);
trs = -trs;

% Trim peaks from the start and troughs from the end of the data
while(pLocs(1) < tLocs(1))
    pLocs(1) = [];
    pks(1) = [];
end

while(tLocs(end) > pLocs(end))
    tLocs(end) = [];
    trs(end) = [];
end

% Calculate arousal metric
peakHeights = pks - trs;
peakWidths = raw_signal_times_secs(pLocs) - raw_signal_times_secs(tLocs);
peakRiseRate = peakHeights ./ peakWidths';
peakValues = peakHeights .* peakRiseRate;

peakEffect = zeros(numel(filtered_signal),1);

peakEffect(pLocs) = peakValues;

arousal_metric = RunningFn(@sum,peakEffect,5000);
\end{verbatim}
	\end{minipage}
	\caption{MATLAB code for computing continuous arousal metric from filtered SC signal.}
	\label{fig:matlabGsrMetric}
\end{figure}