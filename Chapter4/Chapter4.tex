% !TEX root = ../Dissertation.tex
%===================================================================================================

\chapter{Data Collection}




\section{B cells are developing in the NOD thymus}
The reason for B cells being present in the thymus is unknown. 
The potential mechanisms, as outlined in \toref{somewhere in background} are B cells developing from progenitors within the thymus or B cells are being released prematurely from the bone marrow and migrating to the thymus where they continue and complete their development.
In order to investigate the potential for development of B cells within the thymus further, methods such as flow cytometry, magnetic-activated cell sorting (MACS) and polymerase chain reaction (PCR) were used to assess the possibility for intrathymic B cell development.
The process of investigation had two arms.
One involved looking for B cell progenitors and B cells at various stages of development to assess the presence of developing B cells in the thymus, and the other involved looking at the thymic and bone marrow niches for their factors that can influence or drive lymphocyte differentiation and development.

\todo{here, write about the potential for B cells developing in the thymus and the methods used to investigate it. Could say that the methods involved carrying out magnetic cell sorting and this required a lot of optimisation to get best efficiency/yield}

\subsection{Magnetic associated cell sorting optimisation}
\todo{include results for efficiency/yield for both Miltenyi and Qiagen}
Firstly, very early B cell progenitors, termed B cell-biased lymphoid progenitors (BLPs) \toref{background somewhere on BLPs}.
However, in order to look for these in the thymus, it was first necessary to deplete the thymus of any mature cells as the progenitor populations are very small compared to the populations of mature lymphocytes.
The MACS methods used required a lot of optimisation and testing of multiple different bead/magnet kits. 
Firstly, Miltenyi columns and lineage depletion beads were tested to assess their efficiency. 
\fig{add figure showing efficiency of beads/columns here}.
To start with, only one round of depletion was carried out which gave an efficiency of \todo{look at efficiencies}.
This efficiency was improved to \todo{add efficiency of 2 rounds} when two rounds of depletion were carried out. 
However, while this method gave a very pure sample with very few mature cells remaining, the yield was not very good.
Since the point of the depletion was just to reduce the quantity of mature cells so that the progenitor populations could be seen by flow cytometry, it was more important for the experiment that there was a lot of cells to interrogate rather than them being very pure.
Therefore, Qiagen beads \todo{find real name... anti-rat?} were tried and as shown, \fig{figure of Qiagen depletion}, while the purity was not as good as the Miltenyi beads, the yield was much improved.
This was a better balance for the experiment therefore, Qiagen beads were used.

\subsection{Early B cell progenitors are present in the NOD thymus}
\todo{analyse results comparing NOD/KO/B6 data to see if the presence of BLPs is normal in NOD}

\subsection{Pro and Pre B cells are present in the NOD thymus}

\subsection{CD19+ RAG+ B cells are seen in the NOD thymus}


\subsection{B cell development is dependent on the presence of a mature B cell}
\subsubsection{Transfer experiment - Transfer of NOD thymic cells to B cell KO NOD mice}

\subsection{T cell development looks normal/abnormal in NOD mouse compared to control}
\todo{analyse data to see whether it is normal or abnormal. This is a good control to give an indication of the overall condition of the thymus, preferably over time to see how it changes as mouse ages with normal physiological atrophy of the thymus}

\subsection{Conclusion}




\section{There is potentially a population of cells expressing both T and B cell receptors in the NOD thymus}

\subsection{Elevated level of CD8 on CD19+ cells in the thymus}

\subsection{IgM+TcRb+ cell presence}
\todo{include isotype control data}

\subsection{RAG+IgM+TcRb+ cell presence in the NOD thymus}

\subsection{Conclusion}




\section{Potential contribution of thymic B cells to T1D}

\subsection{Pilot study to track movement/homing of thymic B cells on transfer into recipient mice}
\todo{where B cells traffic to could give indication of their role in T1D. Also, use of GFP as  a marker. However, numbers were very small so analysis difficult. Needs to be done again in the future. Comment on whether it may be possible and what would need to change to make it work}
In order to look at the potential impact that thymic B cells may be having on the pathogenesis of T1D, a pilot study looking at the migration of donor thymic B cells following intravenous injection into recipient mice was set up. 
At time points of 7 and 11 days post injection, recipient mice were sacrificed and their thymus, spleen, bone marrow, pancreas, pancreatic lymph node and lateral axillary lymph node were all analysed by flow cytometry.
The tissues to analyse were chosen for the following reasons:
\begin{itemize}
\item Thymus
\item Spleen
\item Bone marrow
\item Pancreas
\item Pancreatic lymph node
\item Lateral axillary lymph node
\end{itemize}

\subsection{Autoantibody production}

\subsection{Conclusion}

