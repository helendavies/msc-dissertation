% !TEX root = ../Dissertation.tex
%===================================================================================================

\chapter{Chapter 4 - Data Collection}




\section{B cells are developing in the NOD thymus}

The reason for B cells being present in the thymus is unknown. 
The potential mechanisms, as outlined in \toref{somewhere in background} are B cells developing from progenitors within the thymus or B cells are being released prematurely from the bone marrow and migrating to the thymus where they continue and complete their development.
Data to date suggests that B cells are developing in the thymus due to the fact pro and pre B cells can be seen there. 
However, there are also other methods for investigating potential intrathymic B cell development.
This project focuses on the following areas of investigation:

\begin{enumerate}
\item Utilisation of flow cytometry to look for pro and pre B cells in the thymus of NOD mice and to see how the state of the NOD thymus compares with other control mice.\toref{relevant subsection}

\item Utilisation of flow cytometry to look for very early progenitors at the initial stages of B cell commitment within the thymus. 
This required the use of magnetic-activated cell sorting (MACS) to deplete thymi of mature cells to reveal the smaller progenitor populations. 
The presence of early B cell progenitors in the thymus could then be compared between NOD mice and controls. \toref{relevant subsection}

\item Utilisation of polymerase chain reaction to investigate the presence of transcription factors and genes driving B cell development in the  thymus. \toref{relevant subsection}
\end{enumerate}

Each will be discussed in the relevant sections below.

\subsection{There are pro and pre B cells present in the NOD thymus}


As mentioned previously, following commitment to the B cell lineage, B cells develop through different developmental stages.
The first is the CD19+CD43+IgM- pro B cell stage.
Following this, IgM heavy chain rearrangement begins and the developing B cells produce an IgM heavy chain. 
They can then progress to the CD19+CD43-IgM- pre B cell stage.
Finally, in the bone marrow, B cells produce a fully functioning B cell receptor (IgM) and are then a CD19+CD43-IgM+ immature B cell.
These immature B cells then move to the spleen to finish their development and become mature B cells.

Interestingly, of the CD19+ cells present in the NOD and B6 mice, although the majority are mature IgM+ B cells, there are also populations of developing B cells at the pro and pre B cell stages.
This gives the impression that B cells are developing within the thymus rather than migrating there and warrants further investigation. \fig{Add figure showing pro and pre B cells in NOD and B6}

\begin{figure}
	\includegraphics[width=\textwidth]{Figures/NODpropre.png}
\end{figure}



\fig{Pro and Pre cells present in the NOD thymus. Show a plot of IgM vs CD43 and label which quadrants are pro and pre.}
\fig{\%s of pro and pre in different ages and strains}
\fig{ratios of pro:pre in thymus vs BM. any difference in different strains?}

\subsection{Developing B cells in the thymus express RAG}


To try and identify whether or not pro and pre B cells are developing within the thymus rather than migrating there, the expression of RAG in these developing B cells was investigated.
To do this, a NOD-RAG-GFP reporter mouse was used as this mouse produces GFP when RAG is being expressed.
GFP shows up on flow cytometry therefore it is a useful tool for investigating B cell development.
RAG expression is crucial during B cell development as it allows the rearrangement of the IgM heavy chain and thus allows the progression from pro to pre B cell stage.

Thymi and bone marrow from NOD-RAG-GFP mice of 4, 7 and 11 weeks of age were used for flow cytometry to look for the present of RAG+CD19+ cells.
Positive RAG expression in CD19+IgM- cells would indicate active development.

The intensity of the RAG signal is important for determining the liklihood of active development.
Since GFP is transcribed at the same time as RAG, it initially has a very high intensity.
However, because GFP is a protein it takes a while to degrade and has a half life of ~54hrs.
This means that only the CD19+RAGhigh population can be taken as actively expressing RAG and therefore GFP.
GFPlow cells will have deactivated RAG and therefore the GFP signal will be decaying.
Setting the gates for RAG\textsuperscript{high} and RAG\textsuperscript{low} populations was done by looking at the RAG levels in developing B cells in the bone marrow.


\begin{figure}
	\begin{subfigure}{\textwidth}
	\includegraphics[width=\textwidth]{Figures/RAGhighlownegthyB.png}
	\caption{Thymic B cells expressing RAG}
	\includegraphics[width=\textwidth]{Figures/RAGhighlownegtotalthy.png}
	\caption{RAG expression in total thymus}
	\end{subfigure}
\end{figure}


\fig{RAG high, low, neg in BM and Thy}
\fig{Graph of change in RAG with age}
\todo{Explanation}


\subsection{B cell development is dependent on the presence of a mature B cell}

Following on from this finding, B cell KO NOD mice thymi were also investigated.
In these mice, as stated in \toref{wherever I talked about KO mice}, these mice are unable to rearrange their IgM heavy chain and therefore cannot progress beyond the pro B cell stage. 
In the bone marrow of these mice, a population of pro B cells is seen, but there are no pre B cells and no IgM+ cells.
It was therefore expected that in their thymus there would also be a population of pro B cells, similar to that seen in the NOD WT mice. 
However, this does not appear to be the case.
As shown in \fig{add graph showing frequencies of pro B cells in NOD, B6 and KO}, it appears that the frequency of pro B cells in the thymus is consistently decreased in B cell KO mice compared to NOD WT mice.
Interestingly, the frequency of IgM-CD43+CD19+ cells in the thymi of KO mice is lower than the frequency seen in the B6 control mouse. 
This gives the impression that a mature B cell is important for allowing the enhancement of pro B cells in the thymus.
Moreover, the need for a mature B cell is more important to increase thymic pro B cells than whatever mechanism is present in the NOD mouse strain that increases the intrathymic pro B cell presence.
The investigation was carried out in a blind trial so that the strains of mice were not known during analysis in order to avoid bias in the results.

\subsubsection{Transfer}

Following on from these findings, a pilot study was set up whereby thymic cells either with or without CD19+ cells from donor NOD WT mice were injected intravenously into NOD KO mice.
This was to see if the presence of thymic cells had any effect on the presence of pro B cells in the KO thymus.

Due to the preliminary nature of this transfer experiment, very small numbers of mice were used.
2 donor mice were taken and split into CD19+ and CD19- fractions and the fractions were then pooled.
4 recipient mice were used, 2 received CD19+ fraction cells and 2 received CD19- fraction cells.
One recipient each for each fraction were taken at 7 days for analysis and the others were taken at 11 days post transfer.
\fig{Show how expt set up - diagram}
\fig{?Table of donor cells - phenotype and number of B cells}

The experiment was of interest due to the fact a similar thing has been attempted before by \citet{Serreze1998}, whereby NOD B cells were injected into NOD KO recipient mice.
Interestingly, it was found that the transferred B cells disappeared in the recipient mice between 6-11 days post transfer.
However, the difference was that the injected B cells were conventional splenic B cells.
Here in my transfer experiment, the transferred cells were of thymic origin and since relatively little is known about thymic B cells, it was wondered whether the outcome would be the same. 
Therefore, the aim of the experiment was 2 fold, one, to see if thymic B cells could survive transfer and two, to see if the presence of thymic B cells had an effect on the frequency of pro B cells in the KO thymus.

Interestingly, in concordance with \citet{Serreze1998}, analysis revealed that mature B cells were only present in the recipient mice at the 7 day post transfer time point.
Further to this, B cells are only seen in the thymus and spleen of recipient mice at the 7 day time point.
No B cells were seen in any of the tissues at the 11 day time point.
This is similar to the findings of \citet{Serreze1998} and could suggest that the B cells are being killed off by cytotoxic T cells which have never developed tolerance to B cells due to their absence during T cell development.
On the other hand, due to the fact that there was only one mouse for each group, the group size is very small and the results must be viewed with caution.

Although B cells are seen in the thymus and spleen of both recipients at the 7 day time point, it seems that there is a higher frequency of CD19+IgM+ cells in the CD19+ recipient.
This finding is not surprising as this recipient received mature B cells in the transferred cells.
However, there are still some B cells seen in the CD19- recipient.
This could suggest two things.
Firstly, B cells are developing from CD19- progenitors which were injected.
Or, secondly, it may be that these B cells were contaminating B cells which were injected in the first place.

\fig{B cells in spleen and thymus of CD19+ and CD19- recipients on 11.6.15 and contrast with lack in spleen and thymus on 15.6.15. Include frequencies to show difference between recipients}

\subsubsection{Is there an effect on the number of pro B cells?}

It was then wondered whether or not the recipient KO mice had an increase in pro B cells in the thymus due to the effect of transferred B cells.
However, there was no difference in the frequency of pro B cells between the KO control and either recipient.
This suggests that either the B cells required to increase the pro B cell population have to be endogenous to the mouse, or it may have to not be of thymic origin, or it may be that during transfer they lose the ability to help increase pro B cells, or it may be there weren't enough of them, or they may have been killed off before they could act... etc!
\fig{Show difference in pro B cell frequencies between NOD control, KO control and recipients (LCs, SCs, CD43+, IgM-, CD19+)}

\todo{Suggest possible ways of carrying on with this area - ie BM chimera etc due to transfer failure, see if can increase pro B cells in KO. ALso, KO B6 may be handy}


\subsection{Early B cell progenitors are present in the NOD thymus}

\subsubsection{Introduction to progenitor work}

Given the discrepency between the NOD mouse and the B6 control mouse in terms of B cell numbers in the thymus, it was decided that investigation into the presence of B cell progenitors in the thymus would be beneficial.
As mentioned in , the progenitor populations giving rise to T and B cells is controversial and there does not appear to be a proven pathway for the development of either.
This leaves a number of questions relating to how these populations can be investigated.
However, literature to date suggests developmental pathways for each and it is this information which has driven the investigation into thymic B cell progenitors.

Of particular interest in this project, information relating to the earliest B cell committed progenitor was sought.
There is much research pertaining to B cell development in the bone marrow and the factors driving the development but it remains a highly complex area.
The matter was complicated further in this project due to the fact B cell development was being investigated in the thymus rather than the bone marrow.
This means that research into this area was based on previous research in the bone marrow and how well B cell development in the bone marrow is mirrored by potential development in the thymus remains unknown.

 However, assuming that B cell development in the thymus was to follow the same pathway as that seen in the bone marrow, the pattern of development should follow the course shown below.

 
\todo{Diagram of B cell development. Show HSCs through MPPs to ALPs/BLPs etc}


 It is not totally clear as to what phenotype the earliest B cell committed progenitor has. 
 However, there is evidence that expression of Ly6D correlates with almost total B cell restriction \toref{Inlay}.
 With this is mind, it was possible to use flow cytometry with an appropriate mix of antibodies to investigate the presence of Ly6D+ progenitors in both the thymus and bone marrow of NOD mouse.
However, Ly6D alone is not sufficient to indicate B cell restriction. 
It is also seen that developing T cells express Ly6D and therefore other markers are required to show which progenitors are likely to B cell restricted rather than T cell restricted.
The markers used alongside Ly6D therefore were Sca-1, c-kit, IL-7Ra and Flt3. 
The reasons for using each of these antibodies are outlined below.
\begin{itemize}
\item Sca-1 - 
\item c-kit - This is expressed at greater levels on T cells compared to B cells therefore in gating of data looking for B cell progenitors, care was taken to not include c-kit high cells as these are likely to B of T cell lineage \toref{Find a reference for this} 
\item IL-7Ra - 
\item Flt3 - The expression of Flt3 on progenitors denotes the retention of B and T cell lineage potential. Therefore, when looking for progenitors that have the potenital to become either a T or B cell, the expression of Flt3 shows that it is not yet committed to the T cell lineage.
\toref{Karsunky2008}
\item Ly6D - \toref{Inlay, Mansson, Zhang2013}
\end{itemize}




The thymus is the site of T cell development and education.
The progenitors for T cells, however, are not totally understood. 
There is belief that T cells originate from CLPs which have migrated from the bone marrow to the thymus where they receive the correct signals which cause them to differentiate into T cells
\subsection{Magnetic-assisted cell sorting optimisation}

Due to the small size of progenitor populations in comparison to mature cell populations in the thymus, it was first necessary to deplete the thymus of the majority of mature cells before looking for progenitors.
For depletion, magnetic-activated cell sorting (MACS) was used.
There are many different kits available and the decision on which to use for the experiments was taken after investigating the efficiency and yield from Miltenyi beads and columnms, and Qiagen anti-rat beads \todo{name of Qiagen beads}

Firstly, Miltenyi columns and lineage depletion beads were tested to assess their efficiency. 
\fig{Miltenyi 1 round. CD4-CD8-, CD19-CD11c-}.
To start with, only one round of depletion was carried out which gave an efficiency of \todo{look at efficiencies}.
This efficiency was improved to \todo{add efficiency of 2 rounds} when two rounds of depletion were carried out \fig{2 round miltenyi lineage depletion. CD4-CD8-, CD19-CD11c-}. 
\begin{figure}
	\begin{subfigure}{\textwidth}
	\includegraphics[width=\textwidth]{Figures/1rounddepletion.png}
	\caption{One round of depletion using Miltenyi lineage depletion beads and columns}
	\end{subfigure}
	\hfill
	\begin{subfigure}{\textwidth}
	\includegraphics[width=0.3\textwidth]{Figures/2rounddepletion.png}
	\caption{Two rounds of depletion using Miltenyi lineage depletion beads and columns}
	\label{subfig:2miltenyi}
	\end{subfigure}
	\hfill
	\begin{subfigure}{\textwidth}
	\includegraphics[width=0.3\textwidth] {Figures/Qiagenbeads.png}
	\caption{Qiagen Beads}
	\label{subfig:Qiagen}
	\end{subfigure}
	
\caption{Figure showing the different methods of depletion. Qiagen beads in shows that 89\% percent of cells are CD4-CD8-}
\end{figure}
However, while this method gave a very pure sample with very few mature cells remaining, the yield was not very good.

Since the point of the depletion was just to reduce the quantity of mature cells so that the progenitor populations could be seen by flow cytometry, it was more important for the experiment that there was a lot of cells to interrogate rather than them being very pure.
Therefore, Qiagen beads were tried and as shown, \fig{figure of Qiagen depletion}, while the purity was not as good as the Miltenyi beads \todo{efficiency of Qiagen beads}, the yield was much improved.
This was a better balance for the experiment, therefore, Qiagen beads were used.
\todo{insert a table showing the efficiencies of each depletion method}


\subsection{BLPs are present in NOD mouse}

\todo{analyse results comparing NOD/KO/B6 data to see if the presence of BLPs is normal in NOD}

Once the method of lineage depletion was decided on, it was then possible to move forward to investigating the early progenitor populations in NOD, NOD KO and B6 thymi.
BLP presence was of particular interest as differences in populations between strains of mice could suggest whether the increase in B cells in the NOD thymus was as a result of processes before or after this point.
BLPs are thought to be the first cell committed to the B cell lineage therefore if the populations of these cells are different, it could suggest abnormalities in factors driving B cell commitment.

\fig{Gating system: LCs, SCs, sca int, kit int, IL-7Ra+Flt3+, Ly6D (+ in BLPs/pre-pro, - in ALPs)}
\fig{?\% of BLPs, ALPs and MPPs in each strain}
\fig{?Ratios of BLP:ALP in BM and thymus}


\subsection{T cell development looks normal/abnormal in NOD mouse compared to control}
\fig{CD4 v CD8 in NOD, KO and B6 (and FVB). Different ages, 5,9,12wks?}
\fig{Graphs of DP, DN and SP}

\todo{analyse data to see whether it is normal or abnormal. This is a good control to give an indication of the overall condition of the thymus, preferably over time to see how it changes as mouse ages with normal physiological atrophy of the thymus}

\subsection{The NOD thymus contains B cell development transcription factors}

If B cells are developing within the thymus, this begs the question as to whether the NOD thymus is providing an environment more conducive to B cell development compared to nondiabetic controls. 
To invesitgate this, primers were designed that were specific for B cell development factors such as E2A, EBF, Pax5, VPreB and CXCL12.
\todo{look in Hagman2006, Welinder2011, Mansson2008, Cobaleda2007, Radtke2013, Riley2013, Tokoyoda2004 for TF info}.

To date, only regular PCR has been carried out to compare whether or not B cell development transcription factors are present in the NOD bone marrow and thymus.
Since normal B cell development occurs in the bone marrow, these transcription factors would be expected there and indeed they were seen there, as shown in \fig{TF gels}. 
Interestingly, all the genes, besides CXCL12, were also present in the NOD thymus too.
This indicates that the thymic environment may also be allowing B cell development.
The lack of CXCL12 in the thymus is not a surprising result as it is a cytokine responsible for holding developing B cells in the bone marrow niche and therefore, may be more bone marrow specific rather than B cell specific.



\fig{regular PCR looking for transcription factors (VPreB, Pax5, EBF, E2A, CXCL12)}

\todo{qPCR of Pax5, Notch, EBF etc}
\fig{qPCR results}

\subsection{Conclusion}

Data to date has so far given evidence that B cells are developing intrathymically in the NOD mouse.
This is shown by pro and pre B cells being present suggesting that developing B cells are present in the thymus.
Further to this, some of these cells show evidence of active RAG expression, indicating that they are actively developing within the thymus rather than having migrated there following premature release from the bone marrow.

There is also evidence to suggest that B lineage committed early progenitors, such as BLPs, are present within the NOD thymus. 
This also gives the impression that B cells are developing from progenitors within the thymus as these are believed to be the progenitors that pro B cells are derived from.
\todo{Add conclusion on differences in frequencies of BLPs in NOD/B6/KO etc to see if any difference}

Finally, it appears that the NOD thymus may be providing factors that are allowing, if not driving, B cell development.
However, it remains to be seen whether this is a normal finding in nondiabetic controls also.
It would also be useful to carry out quantitative PCR of B cell development factors to see the relative levels between the thymus and bone marrow and across strains of mice.
This could suggest whether or not it may be a thymic abnormality in the NOD mouse which is providing a particularly good environment for B cell development compared to other strains of mice.

\section{Some RAG+ cells in the thymus express B and T cell markers}
During analysis of data from previous experiments, it was noted that a large proportion of RAG+CD19+ cells in the NOD thymus, were also CD4+CD8+.
This was of interest and was therefore investigated further as it indicated there is a possibility for cells with markers of both T and B cells.

\subsection{A large proportion of RAG+CD19+ cells in the NOD thymus express CD4 and CD8}

\cref{fig:RAGCD19CD4CD8pos}

RAG acitvation occurs both in developing T and B cells.
It's activation allows the rearrangement of either the T cell receptor (TcR) in T cells or the B cell receptor (IgM).
Flow cytometry of NOD-RAG-GFP reporter mice thymi therefore reveals the RAG activation status indirectly via the production of GFP.
How recently the RAG was activated in a cell can be indicated by the intensity of the GFP signal as GFP degrades over time.
This means that active transcription of RAG/GFP will have a very bright signal, whereas cells which have deactivated RAG will have diminishing levels of GFP and therefore the signal will be less bright.

Interestingly, whilst interrogating NOD thymi for the presence of CD19+RAG+ cells, it was noted that of these cells, a large percentage of them were also CD4+CD8+.
This was an interesting finding as it suggests there may be a point where cells are not yet fully lineage committed and are expressing markers of both T (CD4, CD8) and B (CD19) cells.
These cells all have a high GFP signal and therefore are actively transcribing the protein, indicating that RAG is also active.
This suggests that these cells are expressing markers of both T and B cells as well as rearranging a receptor.

Unfortunately, due to lack of control mice of the correct age, it is not possible to say whether or not this is a normal pysiological process seen in nondiabetic control mice as well as NOD mice or whether it is unique to the NOD.
However, due to normal B cell development occurring in the bone marrow, the thymus and bone marrow were compared for presence of RAG+CD19+CD4+CD8+ cells.
As shown in \fig{figure of BM lack of RAGCD19CD4CD8 pos cells}, none of these cells were seen in the bone marrow. 
This gives the impression that they are not a normal part of B cell development.

This finding could suggest an alternative mechanism by which B cells are increased in the NOD thymus, and that is that a cell is beginning to commit to the T cell lineage, then at some point receiving a signal which is causing it to begin to switch lineages and start expressing B cell markers.
This potential redifferentiation could explain the existence of both B and T cell markers on the cells. 
However, further research would be required to explore this further. \todo{think of expt}.

The gating pattern used to look for these cells in the thymus and bone marrow of NOD WT and NOD KO is shown in \cref{fig:RAGposDP}.
However, it was not known if this is normal during B cell development, therefore the bone marrow was used as a comparison to see if these cells existed there too, where B cells develop normally.
Interestingly, there were no RAG+CD19+CD4+CD8+ cells seen in the bone marrow at all. 
The comparison between a representative 4 week old NOD thymus and bone marrow is shown in \cref{subfig:BMvsThyRAGCD19DP}.
The presence of these RAG+CD19+CD4+CD8+ cells in the thymus was consistent and seen in 4 week and 7 week old NOD thymi with group sizes of 4 or 5 mice in each age group. 
The lack of these cells in the bone marrow suggests that they may be thymus-specific.
However, whether or not the presence of these cells has any relation to the increased population of thymic B cells seen in the NOD mouse remains to be determined.
One way of investigating this is to use non diabetic, control B6 mice.


These cells are present in NOD mice at both 4 and 7 weeks of age and it was therefore investigated whether or not there was a significant difference in the percentages of these cells between the two age groups. 
The bone marrow is also included in the figure as a comparison.
The results of this comparison are shown in \fig{add a graph showing how the ages affect the presence of RAG+CD19+DP cells in the thymus and bone marrow}
\todo{add some data/graph/statistics to show significance, whether there is a difference between the ages etc}


\begin{figure}	
	\begin{subfigure}{\textwidth}
		\includegraphics[width=\textwidth] {Figures/CD19+RAG+DPgating.png}
		\caption{Figure showing the gating pattern when looking for RAG+CD19+CD4+CD8+ cells in a NOD thymus.}
		\label{fig:RAGposDP}
	\end{subfigure}
	\begin{subfigure}{\textwidth}
	\includegraphics[width=0.5\textwidth]{Figures/Thymus1RAGCD19DP.png}
	\includegraphics[width=0.5\textwidth]{Figures/BM1RAGCD19DP.png}
	\caption{Representative plots to show the difference in presence of CD19+RAG+CD4+CD8+ cells in the NOD thymus and bone marrow}
	\label{subfig:BMvsThyRAGCD19DP}
	\end{subfigure}
\caption{This figure is about RAG+CD19+CD4+CD8+ cells in the NOD thymus of 4 and 7 week old NOD mice}
\label{fig:RAGCD19CD4CD8pos}
\end{figure}

\fig{Gating pattern}
\fig{Thymic RAG+CD19+CD4+CD8+ cells}
\fig{BM RAG+CD19+CD4+CD8+ cells}
\fig{Graph showing the difference between 4 and 7 weeks of cells above}



\subsection{RAG+IgM+TcRb+ cell presence in the NOD thymus}

It was then also investigated as to whether the developing cells expressing RAG, CD19, CD4 and CD8 could potentially progress to being RAG+CD19+IgM+TcR+ and therefore express mature T and B cell markers.
Data for this is shown in figure \cref{fig:RAGIgMTcRpos}

\begin{figure}
	\begin{subfigure}{\textwidth}
	\includegraphics[width=0.5\textwidth]{Figures/Thy3RAGIgMTcR.png}
	\includegraphics[width=0.5\textwidth]{Figures/BMRAGIgMTcR.png}
	\caption{Representative thymus and bone marrow analysis looking for RAG+IgM+TcR+ cells}
	\label{subfig:BMvThyRAGIgMTcR}
	\end{subfigure}
\caption{This figure is about RAG+IgM+TcR+ cells in the NOD BM and thymus. 5 thymi and one bone marrow were analysed from 5 11week old NOD mice.}
\label{fig:RAGIgMTcRpos}
\end{figure}

\fig{Graph showing frequencies/numbers of cells}

\subsection{IgM+TcRb+ cell presence}

Whether or not there are mature cells present in other tissues in the body that are expressing TcR and IgM was then explored.
To do this, flow cytometry was carried out on thymic samples from NOD, NOD KO and B6 mice using antibodies specific to TcRb and IgM.
This time, the spleen was also investigated alongside the thymus and bone marrow to provide another comparison and to widen the search for IgM+TcR+ cells irrelevant of RAG expression.
Mice of various different ages were investigated and the results are shown in \fig{figure showing TcR+IgM+ cells (or not!)} .
Isotype controls were also carried out to help with determining whether cells were expressing TcR or IgM or not.
The data from the isotype controls in shown in \fig{isotype controls}.

\fig{Spleen, BM, Thymus of IgM+TcR+ cells. NOD v B6 (?Frozen cells)}


\subsection{Conclusion}
Following on from the flow cytometric data acquired so far relating to the presence of IgM+TcR+ cells in the NOD thymus, it appears that these cells do not exist. 
It may be that the RAG+CD19+CD4+CD8+ cells seen in the thymus are early cells developing in the thymus which are not yet totally committed to one lineage and are expressing markers of multiple lineages, hence the T and B cell markers.
However, it would be beneficial to carry out investigation at the protein level, for example, Western Blotting, in order to see if more mature cells are expressing T and B cell receptors.




\section{Potential contribution of thymic B cells to T1D}

\subsection{Pilot study to track movement/homing of thymic B cells on transfer into recipient mice}

\todo{where B cells traffic to could give indication of their role in T1D. Also, use of GFP as  a marker. However, numbers were very small so analysis difficult. Needs to be done again in the future. Comment on whether it may be possible and what would need to change to make it work.}


As little is known about the function of thymic B cells, a pilot study was set up with the aim of investigating the migration/homing abilities of thymic cells, including B cells when injected intravenously into recipient mice.
The aims of the experiments were to see whether B cells and their progenitors derived from a donor NOD thymus had a preference for migrating back to the thymus following injection or whether they went elsewhere.
By understanding the movements of thymic B cells and their preferential home tissues, it may give an indication as to what, if any, their contribution is to T1D pathogenesis.

\fig{Showing the set up of expt} shows how the experiments were conducted. 
Two donor thymi were taken and split into CD19+ and CD19- fractions using MACS.
CD19+ fractions from both donor were then pooled and the same was done for the CD19- fractions.
Four recipient mice were then injected with either CD19+ or CD19- thymic cells from the donors.
After specified time points, the recipient mice were sacrificed and their tissues were analysed.
The tissues taken for analysis were as follows:
\begin{itemize}
\item Thymus - The origin of the donor cells. It was analysed in the recipient to see if the donor cells would migrate to the recipient thymus giving the impression that thymic B cells have specific properties which allow them to traffic back to the thymus.
For example, they may have specific cytokine sensitivity which results in being able to home back to the thymus, even following intravenous transfer.
\item Spleen - The spleen was included for analysis as this is the normal site of B cell maturation.
Therefore it was of interest to see if thymic derived donor cells would move here like conventional B cells do to finish development.
\item Bone marrow - The bone marrow is the normal site of B cell development and therefore was included to see if the developing donor B cells would preferentially migrate here to develop in the same way as conventional B cells
\item Pancreas - The pancreas contains the Islets of Langerhans which are destroyed in T1D through T and B cell mediated attack.
It is the site of infiltration therefore it was included in analysis to see if thymic B cells were part of the infiltration process.
If so, it may indicate that thymic B cells could have a direct role in the pathogenesis of the disease.
\item Pancreatic lymph node - The PLN is where antigen-presenting cells move to from the pancreas in order to activate T cells.
In T1D, APCs move from the pancreatic islets to the PLN carrying islet antigens and therefore the PLN is in an inflammatory state.
This tissue was therefore included to look at the status of the lymph node to compare it to the control axillary lymph node.
\item Lateral axillary lymph node - This tissue was analysed as a control to compare to the PLN.
Whereas the PLN should be inflamed in T1D, the lateral axillary lymph node should not and can therefore show that any inflammation in the PLN is localised and is not as a result of inflammation throughout the body.
\end{itemize}

\fig{Table of donor cells - numbers/frequencies}



\todo{Questions to ask of data: 1) Where do B cells go? 2)Do the B cells survive? 3) Is GFP a good tracker? 4) Do you get B cells in the CD19- recipients? 5) DO CD19+ cells increase in KOs after transfer? Can't tell as can't differentiate between donor and recipient) 6) Is there new GFP activation after transfer?}

\subsubsection{NOD-RAG-GFP donor thymic cells, NOD WT recipients}

The aim of this experiment was to see where development of B cells occurs in the recipient cells and to see whether or not the presence/absence of CD19+ cells has on the ability of early CD19- progenitor B cells to develop.
It was also an opportunity to see whether GFP is a useful tool for tracking transferred cells or whether the half life is not long enough to track transferred cells for a useful amount of time in the recipient.

\paragraph{Are there any GFP+ cells seen in the recipients?}

\fig{Presence of GFP+ cells}
\fig{BM, Spln and Thy, CD19vIgM, GFP, both time points}
Analysis of the recipient mice at both 4 and 7 days post transfer revealed that there are GFP+ cells seen in the recipient mice at both time points.
In particular, GFP+ cells at both time points seem to migrate to the bone marrow, spleen and control lymph node. 
Some cells migrate to the thymus but only in the CD19- recipients and the population seen there is of very low frequency.

Analysis was mainly concerned with GFP+CD19+IgM+ cell presence.
These cells were looked for in each of the recipient tissues mentioned above.
However, only some tissues seemed to differ significantly from the control and these were the spleen, thymus and bone marrow.
All the other tissues appeared to be very similar to the control NOD mouse included for analysis.

Bone Marrow

4 days:
In this tissue, the presence of CD19+ cells is increased in the CD19+ recipient and the CD19- recipient in comparison to the NOD WT control.
This increase is further accentuated in the CD19+ recipient.
This could suggest that transferred CD19+ cells are moving to the bone marrow.
Contaminating CD19+ cells in the CD19- fraction could account for the small increase in CD19+ population in the CD19- recipient.
However, in terms of GFP presence, it seems that the levels of GFP in the B cells is negligable compared to the control WT NOD mouse.
This suggests that the trasferred cells that may be increasing the CD19+ populations in the bone marrow are mature prior to entering the recipient bone marrow, shown by the lack of GFP.

There are also no more CD19+RAG+ cells in the recipient mice compared to the control mice suggesting that there are no actively developing transferred B cells in the bone marrow.
This could indicate that either B cells are not able to develop following transfer or that they are not doing so in the bone marrow 4 days post transfer.

7 days:
After 7 days, CD19+ cells are increased in both recipients. 
Unlike at 4 days, the increase is similar in both recipients.
GFP is also increased compared to the NOD control.

Spleen

4 days:
In this tissue, a similar picture is seen compared to the bone marrow. 
Again, CD19+ cells are increased in both the recipients compared to the NOD control but the GFP is only slightly increased, if at all.

7 days:
Again at 7 days, the frequency of CD19+RAG+ cells is increased, though the frequency of RAG+ cells in this population is not increased compared to the control.


Thymus

4 days:
In this tissue the frequency of CD19+ cells is increased in the CD19- recipient and decreased in the CD19+ recipient.
There is more GFP in both recipients than the NOD but the most is in the CD19+ recipient.

7 days:
Very similar picture to the 4 day time point.
However, the frequency of B cells is similar in both recipients and both similar/lower than the frequency in the control.
The levels of GFP are increased though in both recipients.


The pancreas, control and pancreatic LNs showed no difference to the control NOD mouse suggesting that they were affected very little by the transferred cells. 
The only difference was an increase in B cells in the control lymph node at the 7 day time point, but no increase in GFP+ cells.

In all the 7 day tissues there were no IgM+TcR+ cells



However, due to the very small scale of the experiment, the results need to be viewed with caution.
Further repeats on a larger scale would need to be carried out to increase the validity of the results.

\subsection{Autoantibody production}
\todo{analyse the FACS of the stroma to look at the purity of stromal cells that will have been put on the microscope slide}
\fig{Representative FACS showing stromal purity}
\fig{Pictures of histology showing serum treated and control PBS treated slides}

\begin{figure}
	\begin{subfigure}{\textwidth}
		\includegraphics[width=\textwidth]{Figures/WTserum2.jpg}
		\caption{Serum treated WT NOD thymic stromal cells}
	\end{subfigure}
\end{figure}

\subsection{Conclusion}


