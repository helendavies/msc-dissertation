% !TEX root = ../Dissertation.tex
%===================================================================================================

\newevenside
\begin{center}
\fontsize{29}{29}\scshape Abstract\\[1cm]
\addcontentsline{toc}{chapter}{Abstract}
\end{center}

B cells are implicated in the pathogenesis of type 1 diabetes (T1D), a serious autoimmune disease. It has been found that in nonobese diabetic (NOD) mice, a well described animal model of T1D, there is an enlarged thymic B cell population, not seen in non diabetic B6 mice, and such B cells may contribute to T1D development. The mechanisms by which B cells are populating the NOD thymus are unknown and delineating the mechanisms would offer new insights into abnormalities of the NOD thymus that may contribute to T1D.

This project investigated potential intrathymic B cell development of B cells in NOD mice, compared to B6 mice. Molecular biological approaches determined that transcripts of essential B cell development transcription factors were present in the NOD thymus. Flow cytometric studies established the presence of thymic pro and pre B cells, as well as thymic CD19\textsuperscript{+} cells expressing RAG enzymes, indicating active rearrangement of the B cell receptor that is crucial for pro to pre B cell progression.
Interestingly, comparative studies using B cell sufficient and deficient NOD mice revealed that mature B cells appear to play an important role in the enhancement of the thymic pro B cell population frequency. 
In contrast to pro B cells, the frequency of the earliest B cell progenitors in the NOD thymus was independent of mature B cells. B cell committed progenitors were decreased in the NOD thymus compared to the B6 thymus, suggesting that there may be an alternative B cell development pathway in the thymus compared to the bone marrow.

The data presented here provides a novel insight into the NOD thymus at the level of thymic B cells. Future studies to establish the definitive mechanisms for thymic B cell development, and their role in the T1D process will be informative.


