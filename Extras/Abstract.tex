% !TEX root = ../Dissertation.tex
%===================================================================================================

\newevenside
\begin{center}
\fontsize{29}{29}\scshape Abstract\\[1cm]
\end{center}

B cells have been implicated in the pathogenesis of type 1 diabetes, a serious autoimmune disease. It has since been found that in nonobese diabetic (NOD) mouse, a well described animal model of T1D, there is an enlarged population of B cells in their thymus which is not seen in non diabetic B6 control mice. The mechanisms by which these B cells are populating the NOD thymus are unknown but it is believed that they are developing intrathymically.
This project aimed to more fully investigate the potential intrathymic B cell development in the NOD thymus and compare different aspects of development with the B6 thymus. Here, evidence to support intrathymic B cell development is presented, showing the thymic presence of developing pro and pre B cells, B cell development transcription factors and CD19\textsuperscript{+} cells expressing RAG enzymes suggesting active B cell receptor rearrangement, a crucial step to allow progression from the pro to pre B cell stage. Interestingly, mature B cells appear to play an important role in the enhancement of the thymic pro B cell population frequency. This was shown by a significant decrease in thymic pro B cell frequency in B cell knockout NOD mice compared to B cell sufficient NOD mice.
Interestingly, investigation into early progenitors was less convincing, suggesting that B cell committed progenitors were decreased in the NOD thymus compared to the B6 thymus, suggesting that there may be an alternative B cell development pathway in the thymus compared to that seen in the bone marrow. Alternative mechanisms are outside the remit of this project, however, the finding of cells expressing B and T cell markers could suggest that there are cells which are producing B and/or T cells in an unconventional manner and could therefore be implicated in excess thymic B cell development.

