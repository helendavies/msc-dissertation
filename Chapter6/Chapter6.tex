% !TEX root = ../Dissertation.tex
%===================================================================================================

\chapter{Conclusion}

This project has investigated some main aspects of B cell develop in order to try and determine whether or not B cells are developing within the thymus.
It appears, through the presence of developing pro and pre B cells and expression of RAG on CD19\textsuperscript{+} cells in the thymus, B cell development may well be occurring intrathymically.
This would also fit with the current literature by the likes of \citet{Akashi2000} and \citet{Perera2013}.
However, it may not be that the B cells are developing in the thymus solely by the same pathway as the B cell development seen in the bone marrow.
It appears that BLPs may be a mechanism for producing B cells in the thymus shown by their presence in the B6 thymus.
However, the increased thymic B cells seen in the NOD, despite a significant decrease in BLPs seen in the NOD thymus compared to B6, suggests that an increase in progenitors cannot account for the increase in thymic B cells.
Instead, it may be possible that B cells are developing from other mechanisms.
One such mechanism may be the dedifferentiation of T cells at some point along their developmental pathway.

The role of the thymic B cell remains to be determined.
However, through the use of a transfer study, evidence is given that thymic B cells preferentially migrate to the spleen.
This may be a normal part of development, similar to that seen in conventional B cell development, or it may be that thymic B cells have a role to play once they arrive in the spleen.

It is not yet possible to know whether the presence of autoantibodies in the NOD thymus is attributable to thymic B cells, this remains to be determined, along with the autoantigens that the autoantibodies are responsive to.


%\section{Contributions}
%\todo{for each question write a bit about how the findings contribute to science}
%\subsection{Research question 1}
%\subsection{Research question 2}
%\subsection{Research question 3}

\section{Limitations and Future Work}

In the future there are a number of things that need to happen to take this project further.
Firstly, it is important that the characterisation of the thymic B cell development process in the NOD can be compared entirely to that seen in the NOD mouse.
The presence of robust controls is vitally important in drawing conclusions from data.
In particular, it would be interesting to know the frequency of CD19\textsuperscript{+}RAG\textsuperscript{+} cells in the NOD thymus to see whether or not the finding in the NOD mouse is unique.
The same can be said for the RAG\textsuperscript{+}CD19\textsuperscript{+}CD4\textsuperscript{+}CD8\textsuperscript{+} cell presence in the thymus.

It would also be useful to determine the role of a mature B cell in the enhancement of the pro B cell population in the NOD mouse.
This would be possible using the methods of bone marrow replacement supplemented with B cells mentioned above.

One of the main things to investigate in light of this project is the potential mechanisms of B cell development.
Before ruling out the BLP as the thymic B cell progenitor, it would be necessary to look in younger and older mice to see if the variability in marker expression could be reduced.
Alongside this, it would also be necessary to look at other potential B cell development mechanisms, for example the potential for B cell development from T cells.
To investigate the potential for redifferentiation of B or T cells, looking for relative levels of transcription factors between B6 and NOD thymi could prove beneficial as it may show an imbalance which could account for the excess B cells in the NOD thymus.

The action of thymic B cells also remains to be determined. 
Whilst preliminary data has been obtained for the migrating ability of transferred thymic B cells, it would be interesting to look in more detail at the homing of the progenitors to see whether they specifically require the thymic environment to develop, making them distinct from conventional B cells.
It would also be benficial to determine whether or not thymic B cells are contributing to negative selection in the thymic, either in a beneficial way or not.

\section{Concluding remarks}
\todo{write a bit about what I did... again.}



	
	
	
	
	
	
	