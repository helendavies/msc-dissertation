% !TEX root = ../Dissertation.tex
%===================================================================================================

\chapter{Introduction}
\label{sec:introduction}

\todo{Write a short introduction and explain what the problem is}

\section{Approach}

\todo{Explain briefly how I am going to approach the problem}

\section{Hypotheses}
\todo{Write about the approaches I am going to use to tackle the problem}
\section{Contributions}
\todo{write about how my research will contribute to the field of T1D}

\section{Structure}
This dissertation is divided into 6 chapters, described below.
\begin{itemize}
\item Chapter 1: Introduction - Introduction to the project including a brief overview of aims and potential contribution to field of T1D research 
\item Chapter 2: Background - Previous work relating to the project to give an understanding of the current state of the research field
\item Chapter 3: Methods - Materials, techniques and data analysis methods used in the project
\item Chapter 4: Data Collection - Overview of the research question being posed and the results obtained from experiments designed to tackle the question
\item Chapter 5: Discussion - Potential meaning of the results obtained and how they could be contributing to the overall T1D pathogenesis
\item Chapter 6: Conclusion - Conclusions that can be drawn from the data obtained and future work that could be carried out to pursue the research questions further
\end{itemize}
