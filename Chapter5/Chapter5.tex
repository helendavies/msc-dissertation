% !TEX root = ../Dissertation.tex
%===================================================================================================

\chapter{Chapter 5 Title}

\Cref{fig:gtaSchematic} is another nice tikz diagram.

\pgfdeclarelayer{background}
\pgfdeclarelayer{middle}
\pgfsetlayers{background,middle,main}

\begin{figure}[h]
	\centering
	%\tikzset{external/remake next}	
	\tikzsetnextfilename{gtaSchematic}
	\begin{tikzpicture}

		
		\node (gta) [align=center, blue!50!black,text=black] {Simulation Software\\(GTA)};
		
		\node (script) [below=0.5cm of gta.south, font=\small\sffamily, draw, anchor=north, rounded corners, blue!50!black, fill=blue!5, text=black] {Custom "mission" script};
		
		\begin{pgfonlayer}{middle}
			\node (simulator) [draw, fit={(gta) (script)},inner sep=0.5cm, rounded corners, blue!50!black, fill=blue!2, text=black] {};
		\end{pgfonlayer}
		
		\node (software) [below=2cm of simulator.south, align=center, draw, rounded corners, draw=blue!50!black, text=black, fill=blue!5] {Simulation monitoring \\ and control software};
		
						
		\draw [blue!50!black,line width=1.2pt,decoration={markings, mark=at position 0.4 with {\arrow[scale=1.5]{>}}},postaction=decorate] 
			(script.east) to [out=0, in=0] 
			node [black,auto,pos=0.5,font=\smaller\sffamily,text width=3cm, anchor=south west] {\begin{itemize}\item[$\bullet$] Road position \item[$\bullet$] Heading \item[$\bullet$] Speed \end{itemize}}
			(software.east);
		
		\draw [blue!50!black,line width=1.2pt,decoration={markings, mark=at position 0.4 with {\arrow[scale=1.5]{>}}},postaction=decorate] 
			(software.west) to [out=180, in=180] 
			node [black,auto,pos=0.5,swap,font=\smaller\sffamily,text width=2.5cm,anchor=north east] {\begin{itemize}\item[$\bullet$] Weather \item[$\bullet$] Traffic density \item[$\bullet$] Scenario reset \end{itemize}}
			(script.west);
			
		\coordinate (sideWidth) at (4,0);
		\coordinate (lineLeft) at ($(simulator.south west)!0.5!(software.north west) - (sideWidth)$);
		\coordinate (lineRight) at ($(simulator.south east)!0.5!(software.north east) + (sideWidth)$);
		
		\draw [blue!50!black, line width=2pt, dashed]
			(lineLeft)
			to node[auto,font=\smaller\sffamily] {Network interface} 
			(lineRight);
		
		\begin{pgfonlayer}{background}
			\node [inner sep=0pt,fill=blue!5,fit={(lineLeft) (lineRight) ($(simulator.north) + (0,1)$)}] {};
			\node [inner sep=0pt,fill=blue!10,fit={(lineLeft) (lineRight) ($(software.south) - (0,1)$)}] {};
			\node [inner sep=0pt,line width=10pt,draw=white,rounded corners=10pt,fit={(lineLeft) (lineRight) ($(software.south) - (0,1)$) ($(simulator.north) + (0,1)$)}] {};
		\end{pgfonlayer}
		

	\end{tikzpicture}
	\caption{Driving simulator monitoring and control schematic}
	\label{fig:gtaSchematic}
\end{figure}

You might want to show multiple aligned plots. See \Cref{fig:allGtaEda}.

\begin{figure}
	\centering
	%\tikzset{external/remake next}		
	\tikzsetnextfilename{allGtaEda}
	\begin{tikzpicture}[trim axis group left]
		
		\begin{groupplot}[
			group style={
				group size=3 by 4,
				%vertical sep=1cm,
				xlabels at=edge bottom,
				ylabels at=edge left,
				},
			width=0.35\textwidth,
			height=0.35\textwidth,
			%ymin=40,
			%ymax=100,
			ylabel=Skin conductance,
			ytick=\empty,
			xlabel=Time (seconds),
			axis y line=left,
			axis x line=bottom,
			enlarge x limits=0.05,
			enlarge y limits=0.05,
			title style={anchor=north},
			tick label style={font=\scriptsize\sffamily},
			label style={font=\scriptsize\sffamily},
			title style={font=\scriptsize\sffamily},
			/tikz/line width=0.6pt,
			/tikz/line join=round,
			/tikz/line start=round,
			/tikz/line end=round,
			]
			
			\nextgroupplot[title=Subject 1]
			
			\addplot[blue] file {Data/gtaEDA/S1easyEda.csv};
			\addplot[red] file {Data/gtaEDA/S1hardEda.csv};
			
			\nextgroupplot[title=Subject 2]
			
			\addplot[blue] file {Data/gtaEDA/S2easyEda.csv};
			\addplot[red] file {Data/gtaEDA/S2hardEda.csv};
			
			\nextgroupplot[title=Subject 3]
			
			\addplot[blue] file {Data/gtaEDA/S15easyEda.csv};
			\addplot[red] file {Data/gtaEDA/S15hardEda.csv};
			
			\nextgroupplot[title=Subject 4]
			
			\addplot[blue] file {Data/gtaEDA/S4easyEda.csv};
			\addplot[red] file {Data/gtaEDA/S4hardEda.csv};
			
			\nextgroupplot[title=Subject 5]
			
			\addplot[blue] file {Data/gtaEDA/S5easyEda.csv};
			\addplot[red] file {Data/gtaEDA/S5hardEda.csv};
			
			\nextgroupplot[title=Subject 6]
			
			\addplot[blue] file {Data/gtaEDA/S6easyEda.csv};
			\addplot[red] file {Data/gtaEDA/S6hardEda.csv};
			
			\nextgroupplot[title=Subject 7]
			
			\addplot[blue] file {Data/gtaEDA/S8easyEda.csv};
			\addplot[red] file {Data/gtaEDA/S8hardEda.csv};
			
			\nextgroupplot[title=Subject 8]
			
			\addplot[blue] file {Data/gtaEDA/S9easyEda.csv};
			\addplot[red] file {Data/gtaEDA/S9hardEda.csv};
			
			\nextgroupplot[title=Subject 9]
			
			\addplot[blue] file {Data/gtaEDA/S11easyEda.csv};
			\addplot[red] file {Data/gtaEDA/S11hardEda.csv};
			
			\nextgroupplot[title=Subject 10]
			
			\addplot[blue] file {Data/gtaEDA/S12easyEda.csv};
			\addplot[red] file {Data/gtaEDA/S12hardEda.csv};
			
			\nextgroupplot[title=Subject 11]
			
			\addplot[blue] file {Data/gtaEDA/S13easyEda.csv};
			\addplot[red] file {Data/gtaEDA/S13hardEda.csv};
			
			\nextgroupplot[title=Subject 12]
			
			\addplot[blue] file {Data/gtaEDA/S14easyEda.csv};
			\addplot[red] file {Data/gtaEDA/S14hardEda.csv};
			
							
		\end{groupplot}
		

	\end{tikzpicture}
	\caption[Band-passed skin conductance for all subjects]{Band-passed skin conductance for all subjects in {\color{blue}easy (1)} and {\color{red}hard (6)} conditions. Missing data for subjects 1 and 2 is due to hardware failure.}
	\label{fig:allGtaEda}
\end{figure}


\subsubsection*{A cool diagram}

\Cref{fig:sampleEOG} is a particularly cool use of group plots with lines between coordinate systems...

\begin{figure}
	\centering
	%\tikzset{external/remake next}		
	\tikzsetnextfilename{sampleEOG}
	\begin{tikzpicture}[trim axis left]
		
		\begin{groupplot}[
			group style={
				group size=1 by 3,
				vertical sep=2cm,
				},
			width=0.8\textwidth,
			height=0.5\textwidth,
			ytick=\empty,
			ylabel=Voltage,
			xlabel=Time (seconds),
			axis y line=left,
			axis x line=bottom,
			enlarge x limits=0.05,
			enlarge y limits=0.05,
			]
			
			\nextgroupplot[
				title=Raw EOG,
				title style={anchor=north},
				]
			
			\addplot[blue] file {Data/eogRaw.csv};
			
			\nextgroupplot[
				title=Band-Passed EOG,
				title style={anchor=north},
				]
				
			\addplot[blue] file {Data/eogBandPassed.csv};
			\coordinate (ZoomStartA) at (axis cs:20,\pgfkeysvalueof{/pgfplots/ymin});
			\coordinate (ZoomStartB) at (axis cs:36,\pgfkeysvalueof{/pgfplots/ymin});
			\coordinate (ZoomStartATop) at (axis cs:20,\pgfkeysvalueof{/pgfplots/ymax});
			\coordinate (ZoomStartBTop) at (axis cs:36,\pgfkeysvalueof{/pgfplots/ymax});

			\nextgroupplot[
				xmin=21,
				xmax=35,
				]
			
			\addplot[
				blue,
				line width=1pt, 
				line join=round,
				] file {Data/eogSection.csv};
				
			\coordinate (ZoomEndA) at (axis cs:\pgfkeysvalueof{/pgfplots/xmin},\pgfkeysvalueof{/pgfplots/ymax});
			\coordinate (ZoomEndB) at (axis cs:\pgfkeysvalueof{/pgfplots/xmax},\pgfkeysvalueof{/pgfplots/ymax});				
			\coordinate (ZoomEndABottom) at (axis cs:\pgfkeysvalueof{/pgfplots/xmin},\pgfkeysvalueof{/pgfplots/ymin});
			\coordinate (ZoomEndBBottom) at (axis cs:\pgfkeysvalueof{/pgfplots/xmax},\pgfkeysvalueof{/pgfplots/ymin});				
			
			\addplot[
				only marks, 
				red,
				mark=10-pointed star,
				mark size=6pt,
				] file {Data/eogBlinks.csv};
				
		\end{groupplot}
		
		% Draw zoom box
		\begin{pgfonlayer}{background}
			\path[fill=yellow!20,draw=red!50] 
				(ZoomStartA) -- 
				(ZoomStartATop) -- 
				(ZoomStartBTop) -- 
				(ZoomStartB) .. controls ($(ZoomStartB)+(0,-2)$) and ($(ZoomEndB)+(0,1)$)..
				($(ZoomEndB)-(0,1)$) -- 
				(ZoomEndBBottom) -- 
				(ZoomEndABottom) -- 
				(ZoomEndA) .. controls ($(ZoomEndA)+(0,1)$) and ($(ZoomStartA)+(0,-1)$) ..
				(ZoomStartA);

		\end{pgfonlayer}


	\end{tikzpicture}
	\caption[A typical EOG signal]{A typical EOG signal. The raw signal (top) is bandpassed (middle) and then statically thresholded to find blinks (bottom).}
	\label{fig:sampleEOG}
\end{figure}






