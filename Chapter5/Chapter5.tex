% !TEX root = ../Dissertation.tex
%===================================================================================================

\chapter{Discussion}

\section{Intrathymic B cell development}
The aims of this project was to further investigate the potenitial intrathymic development of thymic B cells seen in the NOD mouse.
The increase in B cells seen in the thymus of these mice compared to the thymus of control, nondiabetic B6 mice hs prompted investigation into the mechanism by which B cells arise in the thymus.
The main hypotheses so far are development, where thymic B cells are developing from progenitors in the thymus, and migration, where B cells are being released from the bone marrow and migrating to the thymus.
However, there is overlap between these two hypotheses as, in order to develop from progenitors in the thymus, the progenitors must migrate from the bone marrow to the thymus as the thymus has no self-renewing cells.
Therefore, it is more a question of at what point B cells are at when they develop in the thymus, the first point being lineage commitment and the formation of so-called BLPs, and the end point being mature B cells circulating through the blood and then taking up residence in the thymus.

The general consensus is that B cells are developing within the thymus with very little contribution from the circulation.
Experiments involving parabiotic congenic mice showed that of the thymic B cells, the majority were of the endogenous phenotype, not the parabiotic partner. 
This gave the impression that the thymic B cells have more to do with the thymus itself rather than the circulation \citep{Perera2013}.
Similarly, \citet{Akashi2000} found that when donor B cells of a different Ly5 isoform were injected into recipient mice, very few of the thymic emigrant B cells were of donor origin, suggesting that they could be more attributable to development within the host rather than a result of circulating B cells stopping in the thymus.

Therefore, if B cells are developing within the thymus with little contribution from the circulation, it is of interest to know at what point along the B cell developmental pathway they are beginning their intrathymic development.
To investigate this, B cells were looked at at different developmental stages such as pro and pre B cells.
These stages are following lineage commitment.
However, the stages before were also investigated by looking for early B cell progenitors, thought to be Sca-1\textsuperscript{low}c-kit\textsuperscript{low}Flt3\textsuperscript{+}IL-7R$\alpha$\textsuperscript{+}Ly6D\textsuperscript{+} BLPs.

Pro and pre B cells are seen in the thymus.
However, it is not known whether these are developing or have migrated following premature release from the bone marrow.
To tackle this question, RAG expression was looked for in CD19\textsuperscript{+} cells in the thymus of NOD mice.
It was seen that there are populations of CD19\textsuperscript{+}RAG\textsuperscript{high}, CD19\textsuperscript{+}RAG\textsuperscript{low} and CD19\textsuperscript{+}RAG\textsuperscript{-} cells in the NOD thymus.
The RAG\textsuperscript{high} cells are very likely developing there within the thymus as this very bright signal was equivalent to that seen in developing T cells in the thymus.
This gives the impression that RAG is being actively transcribed in CD19\textsuperscript{+} cells in the thymus, indicating active BcR rearrangement and B cell development.

These RAG expressing CD19\textsuperscript{+} cells were looked for in NOD-RAG-GFP mice of 4, 7 and 11 weeks of age and interestingly, the CD19\textsuperscript{+}RAG\textsuperscript{high} population frequency didn't change, whereas the CD19\textsuperscript{+}RAG\textsuperscript{low} frequency decreased significantly.
This gives the impression that as mice age, there are less newly developed B cells in the thymus, potentially as a result of decreased development.
However, if development does decrease, the same is not seen in the population of B cells in the thymus, therefore, it may be that following initial development, the B cell numbers in the NOD thymus are increased due to proliferation of B cells.

To investigate this further, it would be interesting to look at the same thing in B6 mice which have a normal, small population of thymic B cells and compare both frequencies and absolute numbers between strains.
If the same pattern was seen in the B6 and the NOD, both in terms of frequency and numbers, it could suggest that the increase in B cells occurs following the development stage, for example, proliferation.

As an alternative, it may be that the decrease in CD19\textsuperscript{+}RAG\textsuperscript{low} cells is due to these newly developed B cells leaving the thymus and travelling elsewhere.
In this case, it may be that the thymus is acting like the bone marrow, harbouring B cell development then releasing immature B cells to continue development elsewhere.
The decrease also coincides with disease progression, therefore it may be that thymic B cells can contribute to the process of insulitis and move to the islets.
However, this is just speculation.
It would be beneficial to track thymic cells in order to see whether they stay within the thymus or move elsewhere.

It was also of interest that the presence of pro B cells in the NOD thymus was linked to the presence of a mature B cell.
The frequency of pro B cells was found to be significantly lower in NOD KO thymi compared to NOD WT and B6 controls.
This is of interest as it begs the question as to what the role of a mature B cell might be.
Some potential roles are outlined below:
\begin{itemize}
\item A mature B cell may be helping with the recruitment of B cell progenitors from the bone marrow to the thymus
\item A mature B cell may provide a survival factor for B cell progenitors within the thymus so that they survive to be able to development
\item A mature B cell may kick start the process of B cell transcription factor expression within the thymus so that the thymus can provide an environment conducive to B cell development
\item A mature B cell may provide a survival factor for newly developed B cells so that once they have developed they are able to survive.
It may be that a mature B cell is important for provision of a survival factor for pro B cells in the thymus, kickstarting the expression of B cell transcription factors in the thymus, provision of a survival factor for newly developed B cells 
\end{itemize}

With the results of the NOD KO mouse, it is unlikely that the action of a mature B cell is following the developmental process and providing a survival factor for newly developed B cells.
If this was the case, the population of pro B cells in the thymus of KO mice would be normal.
It is more likely that it's role is earlier in the process, either B cell progenitor recruitment, kickstarting transcription factor expression or a pro B cell survival factor.

It would be interesting to investigate the influence of a mature B cell on pro B cells in NOD KO mice further.
To do this, it may be necessary to irradiate NOD KO mice then replace with NOD KO bone marrow supplemented with NOD B cells to try and avoid the B cells being killed off by CTLs that have not been tolerised to B cells during development \citep{Serreze1998}.
By adding purified mature B cells, it would be interesting to see if this was sufficient to increase pro B cell frequency in the KO thymus.
Following this, it would also be of interest to try replacing the KO bone marrow supplemented with B6 B cells to see if it is a characteristic of a NOD B cell.
It is not known whether the mature B cell needs to be of splenic or thymic origin, therefore these experiments could also test that by separately transferring bone marrow supplemented with B cells from both tissues.

CD43 expression was also found to be of interest in the NOD mouse compared to the B6 control.
It appeared that the expression of CD43 was much higher on mice of NOD background compared to B6.
The reason for this is not known, however, it is possible that CD43 provides an important tool for allowing T cells to move from the blood into lymphoid tissues.
Evidence for this was shown in 1999 by \citet{Mikulowska1999} when an anti-CD43 antibody was used following transfer of NOD splenocytes into NOD/scid recipient mice. 
Normally, this would cause T1D in the recipient mice and this was indeed seen in the control mice which did not receive L11.
However, in mice receiving L11, it was shown that the onset of T1D was significantly delayed.
It was further shown by \citet{Johnson1999} showed that L11 was capable of providing significant protection from T1D onset in NOD mice.

Considering the results showing increased CD43 expression in NOD thymi compared to B6 thymi along with these previous findings regarding anti-CD43 antibodies, it is worth considering whether T1D in the NOD mouse may be related to the levels of CD43 expressed on IgM\textsuperscript{-} cells in the NOD thymus.
It may be that the increased CD43 expression on NOD mice may be allowing the infiltration of T cells into the pancreas, something which is not seen in B6 mice.
It is worth further investigation into this this is replicable when the group size is increased.


Following the findings that pro and pre B cells are probably developing within the thymus, shown by CD19\textsuperscript{+}RAG\textsuperscript{+} cell presence, earlier progenitors were investigated.
These Sca-1\textsuperscript{low}c-kit\textsuperscript{low}Flt3\textsuperscript{+}IL-7R$\alpha$\textsuperscript{+}Ly6D\textsuperscript{+} BLPs are believed to be the point of B cell commitment, therefore if they are present in the NOD mouse, it is possible that all stages of B cell development are occurring in the thymus.

It was interesting when investigating the presence of BLPs that the frequency of BLPs in the Sca-1\textsuperscript{low}c-kit\textsuperscript{low}Flt3\textsuperscript{+}IL-7R$\alpha$\textsuperscript{+} population of the thymus of the NOD mouse, was significantly decreased compared to the B6.

However, before any assumptions can be made on this data, it is first necessary to ask some questions about BLPs.
\begin{itemize}
\item Are BLPs the sole progenitor for B cell development?
\item Is B cell development from BLPs restricted only to the bone marrow?
\item Can the B cell developmental pathway from HSCs through to pro B cell in the bone marrow be applied to B cell development in the thymus?
\end{itemize}

In order to help determine the answers to the questions above, it would necessary to carry out further experiments.
To determine whether B cells can develop from BLPs in the thymus, it would be interesting to culture BLPs on thymic stromal cell lines.
This in vitro approach may be able to indicate whether or not B cells could develop from BLPs in the thymus.

It is unlikely that B cell development from BLPs is restricted only to the bone marrow, shown by the presence of BLPs in the B6 thymus.
These may be the way in which a normal, small population of B cells arises in the B6 thymus.

However, the NOD thymus has a significantly smaller frequency of BLPs in the thymus compared to the B6.
Despite this, NODs have increased thymic B cells suggesting that there are different mechanisms in the NOD mouse which increase thymic B cells that are not present in the B6.

One possibility arises from the finding of RAG\textsuperscript{+}CD19\textsuperscript{+}CD4\textsuperscript{+}CD8\textsuperscript{+} cells in the NOD thymus.
These cells expressing markers of both T and B cells suggest that there may be an extra stage of normal development whereby developing T or B cells express markers of the other.
On the other hand, this process is unlikely to be a normal part of either developmental pathway due to the absence of these cells in the bone marrow (therefore unlikely to be a part of B cell development) and the fact that T cells are committed to the T cell lineage at the DN stage (prior to CD4 and CD8 expression).

It is therefore worth considering whether it is a normally developing T cell which is late to commit to the T cell lineage and is retaining some characteristics of B cells (CD19), or indeed a cell which is developing with characteristics of both T and B cells.
However, this is unlikely due to the very small prevalence of cells in the thymus that have progressed to expressing both a T and B cell receptor and are IgM\textsuperscript{+}TcR\textsuperscript{+}.

Another possibility which could then contribute to the B cell population in the thymus is that these cells are the midpoint of transitioning from T to B cell (or vice versa).
This would account for the lack of IgM\textsuperscript{+}TcR$\beta$\textsuperscript{+} cells, and the expression of both T and B cell markers.
It may also account for the decreased BLPs in the NOD compared to B6 as B cells developing from T cells would not develop originally from BLPs.
To investigate this possibility, it may be of use to look at the relative levels of B and T cell transcription factors Pax5 and Notch1.
Pax5 is thought to repress Notch1 and therefore T cell development.
Notch1 is thought to repress Pax5 and therefore B cell development.
Quantitative PCR would be a way of looking at this and then a comparison could be made between normal B6 mice thymi and NOD mice thymi.
It may that there is an imbalance in the presence of transcription factors which is allowing the development of the wrong lymphocytes, or the dedifferentiation of one to the other.

Another finding when looking for BLPs in the NOD mouse, it was noted that there was a large variation in each frequency of each marker.
This was not seen in the B6 mouse where each mouse looked very similar.
This gives the impression that the NOD background gives a very dysregulated environment in the thymus which may be leading to the variation from mouse to mouse and the differences between NOD mice and B6 controls.
On the other hand, it may be that variation was due to the chosen time point. 
At 6-8 weeks old, the population of B cells in the thymus is increasing rapidly so it may be that the variation seen between mice could be due to some mice being further along the disease process than others.
To look at this possibility it would be necessary to repeat the experiment in younger and older mice.



\section{Role of thymic B cells}















TSPs may be found by looking at cells in BM that express things that could allow homing to the thymus. Could it be that in NOD mouse, the expression of this is abherrent and therefore also found on B cell precursors so that XS B cell precursors end up in the thymus???



