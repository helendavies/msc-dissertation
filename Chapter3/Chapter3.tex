% !TEX root = ../Dissertation.tex
%===================================================================================================

\chapter{Methods}

\section{Materials}

\subsection{Mice}

NOD, NOD.microMT \citep{Kitamura1991} and C57BL/6 (B6) mice were used.
NOD.microMT mice have a mutation in their IgM heavy chain meaning that they cannot rearrange their heavy chain, cannot progress from the pro B to pre B cell stage and therefore do not have any mature B cells.
All mice were housed in specific pathogen free barrier conditions at York University Biological Services Facility. 
All experimental procedures wewre conducted under U.K. Government Home Office guidelines.


\subsection{Antibodies}

\subsection{Flow cytometer}

\section{Techniques}

\subsection{Cell preparation}

\subsubsection{Preparation from frozen tissue}

Frozen cells defrosted quickly and transferred to 5mL 0.5\% BSA.
Samples spun down for three minutes at 1200rpm, supernatant discarded and cells resuspended in 1mL 0.5\% BSA.
Samples spun down again as before then resuspended in 2mL 0.5\% BSA and put in incubator at 37 degrees for 2 hours.
Samples then spun as before and resuspended as appropriate for subsequent procedure.

\subsubsection{Preparation from fresh tissue}
Bone marrow tissue was obtained by flushing of femur and tibia of mouse with 1 x PBS.
Thymi and spleens were homogenised using a seive and syringe plunger.
Suspensions spun down for three minutes at 1200rpm to pellet cells.
Supernatant discarded.
Cells requiring red blood cell lysis were then resuspended in 1mL of water for 6 seconds followed by 1mL 2 x PBS to neutralise toxicity of water.
Lysed samples then spun down again as before, supernatant discarded and resuspended as appropriate for subsequent procedure.

\subsection{Flow cytometry}

Single cell suspensions prepared as above.
Samples treated with anti-CD16 antibody to block Fc receptors to avoid non-specific antibody binding and incubated for at least 10 minutes in the fridge.
Samples then stained using appropriate fluorescently labelled antibodies and incubated for 25 minutes in the dark in the fridge.
Antibodies used at 1:300 dilution apart from CD4 and CD8 which were used at 1:800. 
Samples washed in 3mL PBS and spun down for 3 minutes at 1200rpm, supernatant discarded then cells resuspended in 500 microL PBS to run on FACS machine.

If biotin/streptavidin antibodies used then a second round of staining and washing used to stain with streptavidin.

\subsection{Magnetic cell sorting}

\subsubsection{Miltenyi beads and columns}

Cells prepared as above the depleted following maufacturers instructions for CD19 adn lineage depletion kits.
The only deviations from these instructions was the use of unsupplemented 0.5\% BSA as the buffer, and the addition of anti-CD19 antibody to block Fc receptors before the addition of antibody.

Attempts were made to reuse the columns multiple times which was sucessful, however, storage was an issue for longer periods of time due to the beads going rusty.

\subsubsection{Qiagen beads}

Cells were prepared as above then resuspended in 500 microlitres of 0.5\% BSA then 1:300 dilution of anti-CD16 antibody and incubated in the fridge for 10 minutes to block Fc receptors.
Samples then stained with appropriate purified antibody and incubated in fridge for 20 minutes.
1 mL Qiagen beads taken and 10mL PBS added.
Beads put on magnet and left for about 10 minutes to allow all beads to adhere.
Pasteur pipette used to remove and discard supernatant and beads resuspended in 500 microlitres of PBS.
Beads put back on magnet and beads allowed to adhere again before removing supernatant as before.
A final wash in 500 microlitres PBS performed then beads finally resuspended in 500 microlitres PBS.

Cells washed in 3mL PBS and then spun down for 10minutes at 1200rpm.
Cells then resuspended in the washed beads and incubated in fridge for 15 minutes.

Samples then put on magnet and beads allowed to adhere.
Supernatant removed with pipette as during washing steps.
Supernatant kept and put back on magnet to remove residual beads and supernatant then transferred to fresh tube as depleted sample.
Samples can then be stained for flow cytometry as above.

\subsection{RNA preparation}

Cells were prepared as above then RNA extracted using RNeasy mini kits following the maunufacturers protocol.
A DNAse step was included to remove residual genomic DNA.
The addition of a DNAse step improved the clarity of bands of products on gel electrophoresis following PCR.
RNA quality was tested on the NanoDrop, with the aim of producing RNA with 260/280 and 260/230 values of near to or above 2.0 which indicates a good purity of RNA.

RNA was stored at -20 degrees for short periods of time and at -80 degrees if stored for a prolonged period.


\subsection{Reverse transcription of RNA to cDNA}

RNA was reverse transcribed to cDNA using Invitrogen Superscript II.
RNA taken and heated at 55 degrees for 10 minutes before being placed on ice.
7 microlitres of first buffer (4 microlitres 5 x 1st strand buffer, 2 microlitres 10mM dNTPs, 1 microlitre OligodT 0.5 microM) was added to 10 microlitres of RNA in RNAse-free PCR strips.
Amount of RNA normalised across samples using data from NanoDrop in order to try and keep quantities of cDNA constant across samples.
RNA then made up to 10 microlitres with RNAse-free water.
Samples then incubated at 65 degree for 5 minutes.
3 microlitres of second buffer (1 microlitre 0.1M DTT, 1 microL RNAse out, 1 microL Superscript II) added to each sample then samples incubated at 65 degree for 45-60 minutes to make cDNA.
cDNA stored at -20 degrees until needed.

\subsection{Primer design}

\subsection{Polymerase chain reaction}

\subsection{Gel electrophoresis}

\subsection{Tacman quantitative polymerase chain reaction}
\todo{learn how to, and then do the experiment}

\subsection{Histology}

\section{Data analysis and interpretation}

\subsection{Flow cytometry}

\subsubsection{Data analysis}

\subsection{Tacman stuff...}

\subsection{Statistics}

\subsection{Graph drawing}


