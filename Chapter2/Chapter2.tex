% !TEX root = ../Dissertation.tex
%===================================================================================================

\chapter{Background}


\section{Type 1 Diabetes - Facts and Figures}
\label{sec:sectionLabel}


Diabetes mellitus is a condition affecting the body's ability to control blood sugar levels.
There are two kinds of diabetes mellitus, type 1 and type 2.
Type 1 Diabetes (T1D) is a serious autoimmune condition whereby the body's own immune system destroys the insulin producing beta cells in the pancreatic Islets of Langerhans \citep{Daneman2006}.
Type 2 Diabetes, however, is a condition where the body becomes either resistant to insulin or the insulin producing beta cells become dysfunctional.
Whereas T1D is antoimmune disease with typical onset during adolescence, T2D is more prevalent in older people and is linked to changes in lifestyle \citep{OxClinMed}.
This project is concerned with T1D and it's autoimmune basis.

According to statistics provided by Diabetes UK, diabetes, both T1D and T2D, affects approximately 3.9 million people in the UK, with many more as yet undiagnosed. 
Of all diabetes sufferers, around 10\% have T1D \toref{Diabetes UK 2015}.

Insulin is a vital hormone produced in response to rising blood sugar levels in order to help cells take up glucose in order to lower blood glucose and allow cells to respire and produce energy.
Without it, glucose metabolism is dysregulated and normal homeostasis cannot be maintained without regular glucose monitoring and insulin administratrion via injection or infusion.
Without strict diet and insulin control, the consequences of T1D can be life threatening both in the long and short term.
In the short term, extremes of blood glucose, either too high or too low can lead hyper- or hypoglycaemia which can prove lethal if not treated promptly.
In the longer term, prolonger mild hyperglycaemia caused by T1D can cause micro- and macrovascular complications leading to blindness, neuropathy, kidney failure and a significantly increased risk of cardiovascular disease.
It is therefore vitally important for research into T1D to continue, with the aim of finding a cure.

There are currently two arms of T1D research.
The first involves using diabetic patients.
This has it's limitations due to the lack of availability of tissues.
The other arm requires the use of animal models, particularly the nonobese diabetic (NOD) mouse, which is currently the best available animal model of T1D.
The NOD mouse will be discussed further below.




\section{The Nonobese Diabetic Mouse}

The nonobese (NOD) mouse is the best available animal model of T1D, first developed by Makino et al \toref{Makino et 1980} in 1980.
It is a model that is genetically predisposed to develop T1D by a well characterised pathogenesis over a well defined time course as outlined below and in \toref{figure showing T1D in NOD}.

\begin{enumerate}
\item Sensitisation - the immune system first becomes aware of islet antigen and incorrectly recognises it as foreign
\item Infiltration - cells of the immune system move to the pancreatic islets and infiltrate the tissue. For a while, the environment remains regulated and no damage occurs
\item Insulitis - the immune cells in the islets become activated and cause inflammation
\item Beta cell destruction - activated cytotoxic T cells (CTL) begin to target and destroy beta cells
\end{enumerate}

It is believed that the NOD mouse is a relevant model for T1D research as it is thought the human disease follows a similar pathogenesis.
It has been seen in donated human diabetic tissues that there is infiltration, insulitis and CTL-mediated beta cell destruction.
Each of the stages of T1D development will be discussed further in the relevant sections below.


\subsection{Initiation of T1D}

It is not known what causes T1D.
There are numerous hypotheses as to how the immune system first becomes aware of islet antigens that normally remain hidden.

One such hypothesis is that during normal development, the pancreas undergoes significant remodelling.
This means that as cells die and are replaced, there is a lot of cell debris produced which requires removal by macrophages.
Normally, this is done without activation of T cells and subsequent inflammation and immune response.
However, it may be possible that the turnover of cells may overwhelm macrophages and their ability to clear debris without causing inflammation.
This could therefore release islet antigens and cause activation of T cells following antigen presentation.

\subsection{Infiltration and Insulitis}
\subsection{Beta cell destruction}



\section{Immune System Involvement}

The immune system is a complex system consisting of interlinking pathways that all work together to act as a powerful army to protect the host from bacteria, viruses and other injurious agents.
It is split broadly into two arms, the adaptive immune system, and the innate immune system, each of which will be discussed in more detail below.


\subsection{The innate immune system}
The innate immune system is the very fast acting, non specific arm of the immune system.
It is inspecific and will therefore attack any invading organisms in order to rid the body of potential threat.
It also has no immunological memory, that is, it is incapable of knowing whether or not the body has encountered the organism before and therefore cannot launch an attacked tailored to the particular invasion.

\todo{Include macrophages and DCs due to their importance in T1D}

\subsection{The adaptive immune system}

In contrast to the innate immune system, the adaptive immune system is slower-acting and specific.
The high specificity of the adaptive arm means that pathogens are recognised by cells of the adaptive immune system and therefore the immune response is designed specifically for the invading target.
The adaptive immune system also has immunological memory so it can remember what it has encountered before so that a quicker response can be mounted when the pathogen is encountered for a second time.

\subsubsection{Cells of the adaptive immune system}
\paragraph{B cells}
\paragraph{T cells}

\todo{Say each cell has a role in T1D, discussed below in \toref{Involvement in T1D}}

\todo{antigen presentation and costimulation}

\section{B cells}
\subsection{role}
\subsection{Development - Progenitor to maturity}
\subsection{Bone Marrow Niche}
\subsection{Genes and Transcription Factors driving B cell development}
\subsection{Tolerance}

\section{T cells}
\subsection{Role}
\subsection{Development}
\subsection{The Thymus}
\subsection{Genes and Transcription factors}
\subsection{Tolerance}



\section{B cell involvement in T1D}

\section{Abnormal B cell presence in the NOD mouse thymus}
\subsection{Thymic B cells}
\todo{thymic B cells are increased in the NOD conpared to nondiabetic controls}

It is normal for B cells to be present in the thymus in small numbers \citep{Isaacson1987, Akashi2000} of non diabetic humans \citep{Isaacson1987} and mice \citep{Akashi2000}. 
However, in the NOD mouse, this population of B cells is dramatically increased \citep{OReilly1994}.
This information has been confirmed by previous work carried out in the laboratory where flow cytometry has been carried out on the thymi of NOD mice and non diabetic control B6 mice.
As shown in \fig{Add figure of JV data}, the presence of B cells in the NOD mouse thymi is significantly increased compared to B6 thymi and the difference between the strains increases as the mice age. (Acknowldegements to Jennifer Varian for provision of raw data). \todo{Explain the graph when it is done!}

  \citep{Christensson1998} for MG and Thymic B cells.

\subsubsection{Potential role of thymic B cells in T1D}
\todo{previous data on autoantibody production and antibody presentation by B cells in T1D}


































