% !TEX root = ../Dissertation.tex
%===================================================================================================

\chapter{Background}


\section{Type 1 Diabetes}
\label{sec:sectionLabel}


\subsection{Facts and Figure}
\subsection{T1D in Man}
\subsection{T1D in Mouse}
\subsubsection{The nonobese diabetic mouse}

\section{The Immune System}
\subsection{The adaptive immune system}
\subsubsection{Role of the adaptive immune system}
\subsubsection{Cells of the adaptive immune system}
\subsubsection{Role of the adaptive immune system in type 1 diabetes}
\subsection{The innate immune system}
\subsubsection{Role of the innate immune system}
\subsubsection{Cells of the innate immune system}
\subsubsection{Innate immune system involvement in type 1 diabetes}

\section{Normal lymphocyte development}
\subsection{B cell development}
\subsubsection{Site of development - the bone marrow niche}
\todo{normal B cell development occurs in the bone marrow in a specific niche.}
\subsubsection{Stages of development}
\todo{Include all stages of development from HSC through BLP, Pre-pro, pro, pre and immature B cell. Introduce markers such as Ly6D, kit, sca-1 etc.}
\subsubsection{Genes and transcription factors driving B cell development}
\todo{include transcription factors here.}
\subsubsection{Tolerance}
\todo{central and peripheral tolerance and how these act to help prevent autoimmunity}
\subsection{T cell development}
\subsubsection{Site of development - the thymic niche}
\todo{the thymic microenvironment and how it is geared up for T cell development.}
\subsubsection{Genes and transcription factors driving T cell development}
\todo{Transcription factors}
\subsubsection{Tolerance}

\section{Abnormal B cell presence in the NOD mouse thymus}
\subsection{Thymic B cells}
\todo{thymic B cells are increased in the NOD conpared to nondiabetic controls}

It is normal for B cells to be present in the thymus in small numbers \citep{Isaacson1987, Akashi2000} of non diabetic humans \citep{Isaacson1987} and mice \citep{Akashi2000}. 
However, in the NOD mouse, this population of B cells is dramatically increased \citep{OReilly1994}.
This information has been confirmed by previous work carried out in the laboratory where flow cytometry has been carried out on the thymi of NOD mice and non diabetic control B6 mice.
As shown in \fig{Add figure of JV data}, the presence of B cells in the NOD mouse thymi is significantly increased compared to B6 thymi and the difference between the strains increases as the mice age. (Acknowldegements to Jennifer Varian for provision of raw data). \todo{Explain the graph when it is done!}

\subsubsection{Potential impact of thymic B cells}
\todo{include the fact they are found in other autoimmune diseases too such as MG etc}

Intrathymic B cells in patients with MG \citep{Christensson1998}.

\subsubsection{Potential role of thymic B cells in T1D}
\todo{previous data on autoantibody production and antibody presentation by B cells in T1D}


































