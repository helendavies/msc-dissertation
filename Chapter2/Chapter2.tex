% !TEX root = ../Dissertation.tex
%===================================================================================================

\chapter{Background}


\section{Section Title}
\label{sec:sectionLabel}

\todo{Write this section.}

\subsection{Subsection Title}
You can \toref{mark things as needing a reference}.

\Citet{Vigouroux1879} was the first to observe that...

See \Cref{fig:edaExample} for a sample figure drawn from a data file.


\fig{You need a figure here}.


%\tikzset{external/remake next} % uncomment this line to force the graphic to be regenerated, even if it's cached from last time.
\begin{figure}
	\centering
	\tikzsetnextfilename{edaExample}
	\begin{tikzpicture}[trim axis left,x=1pt,y=1pt]


	\begin{axis}[width=\textwidth,
			   height=0.5\textwidth,
                           ylabel=Skin Conductance,
                           xlabel=Time (minutes),
                           ytick=\empty,
                           axis x line=bottom, 
                           axis y line=left,
                           enlarge x limits=0.05,
			   enlarge y limits=0.05]
	\addplot[color=red, line width=1,line join=round,cap=round] file {Data/edaExample.csv};
	\end{axis}

	\end{tikzpicture}

	\caption{Typical skin conductance data}
	\label{fig:edaExample}
\end{figure}

\Cref{fig:ecg} is a more elaborate plot.

You can \new{mark something as new to draw Peter's attention to it}

%\fig{Sample ECG}
\begin{figure}[h]
	\centering
	%\includegraphics[width = \textwidth]{Figures/ecg.png}
	%\tikzset{external/remake next}
	\tikzsetnextfilename{ecg}
	\begin{tikzpicture}[trim axis left]
		\begin{axis}[width=0.9\textwidth,
			           height=0.5\textwidth,
		                   xlabel=Time (seconds),
		                   ylabel=Measured voltage,
		                   ytick=\empty,
		                   xtick={0,...,5},
		                   xmax=5,
		                   xmin=0,
		                   axis x line=bottom,
		                   axis y line=left,
		                   enlarge x limits=0.05,
		                   enlarge y limits=0.05,
		                   ]
			\addplot[color=blue,
			             line width=1pt,
			             line join=round,
			             cap=round] file {Data/ecg.csv};

			\coordinate (ArrowStart) at (axis cs:1.2,0.23);
			\coordinate (ArrowEnd) at (axis cs:0.73,0.18);	
			\coordinate (ArrowEndd) at (axis cs:1.4,0.18);	
			
		\end{axis}

		\path[draw=black, line width=1pt,->] 
			(ArrowStart) node[anchor=south] {`R' peaks} -- (ArrowEnd);
		\path[draw=black, line width=1pt,->] 
			(ArrowStart) -- (ArrowEndd); 
			

	
	\end{tikzpicture}

	\caption{A typical electrocardiogram, recorded over about 5 seconds.}
	\label{fig:ecg}
\end{figure}

\Cref{tab:eyeMeasurement} is a good example of a nice table.

\renewcommand{\arraystretch}{4.5}
\begin{table}
	\centering
	\small
	\begin{tabu} to \textwidth {  >{\centering}X[1.5,m] || >{\centering}X[m] >{\centering}X[m] >{\centering}X[m] >{\centering\arraybackslash}X[m] }
	%\hline
	\textbf{Method} & \textbf{Hardware} & \textbf{Intrusiveness} & \textbf{Accuracy} & \textbf{Frame rate} \\
	\hline
	Direct observation & Webcam & Low & $\approx 1$\textdegree & $30$ Hz \\
	Photoelectric viewing & Projected light; photosensors & Low & $3'$ over $30$\textdegree & $1000$ Hz \\
	Reflection & Webcam; light source & Low & $0.5$ -- $1$\textdegree & $30$ Hz \\
	EOG & Multi-channel voltmeter & High & Disputed & $1000$ Hz \\
	Electromagnetic & Metal coils attached to eye & Extreme & $5''$ & $1000$ Hz \\
	\hline
	\end{tabu}
	\caption{Summary of eye movement measurement techniques}
	\label{tab:eyeMeasurement}
\end{table}


\subsubsection*{A non-numbered subsubsection}

With some content.

\section{Conclusion}

Conclusion to my background chapter...


































