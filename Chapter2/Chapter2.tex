% !TEX root = ../Dissertation.tex
%===================================================================================================

\chapter{Background}


\section{Type 1 Diabetes - Facts and Figures}
\label{sec:sectionLabel}


Diabetes mellitus is a condition affecting the body's ability to control blood sugar levels.
There are two kinds of diabetes mellitus, type 1 and type 2.
Type 1 Diabetes (T1D) is a serious autoimmune condition whereby the body's own immune system destroys the insulin producing beta cells in the pancreatic Islets of Langerhans \citep{Daneman2006}.
Type 2 Diabetes, however, is a condition where the body becomes either resistant to insulin or the insulin producing beta cells become dysfunctional.
Whereas T1D is an antoimmune disease with typical onset during adolescence, T2D is more prevalent in older people and is linked to changes in lifestyle \citep{OxClinMed}.
This project is concerned with T1D and it's autoimmune basis.

According to statistics provided by Diabetes UK, diabetes, both T1D and T2D, affects approximately 3.9 million people in the UK, with many more as yet undiagnosed. 
Of all diabetes sufferers, around 10\% have T1D \toref{Diabetes UK 2015}.

Insulin is a vitally important hormone involved with the maintenance of normal blood glucose levels.
Insulin is vital for stopping the production of extra glucogen by means of gluconeogenesis, glycogenolysis, lipolysis, ketogenesis and proteolysis.
However, insulin deficiency as seen in T1D means that this cannot happen and also causes the increase in glucagon, the hormone which counteracts insulin to increase blood glucose by the mechanisms stated above\citep{Sonksen2000}.
With effective glucose monitoring and management through injections or infusions of insulin keeps this process in check as there should never be absolute insulin deficiency.

Alongisde the charactistic symptoms of excessive thrist, drinking and urination, there are serious side effects both in the short and longer term.
In the short term, extremes of blood glucose, either too high or too low can lead hyper- or hypoglycaemia which can prove lethal if not treated promptly.
In the longer term, prolonger mild hyperglycaemia caused by T1D can cause micro- and macrovascular complications leading to blindness, neuropathy, kidney failure and a significantly increased risk of cardiovascular disease\citep{OxClinMed}.
It is therefore vitally important for research into T1D to continue, with the aim of finding a cure.

There are currently two arms of T1D research.
The first involves using diabetic patients.
This has it's limitations due to the lack of tissue availability and because a lot of the beta cell destruction occurs prior to the appearance of T1D symptoms \citep{Thomas2000}.
The other arm requires the use of animal models, particularly the nonobese diabetic (NOD) mouse, which is currently the best available animal model of T1D.
The NOD mouse will be discussed further below.




\section{The Nonobese Diabetic Mouse}

The nonobese (NOD) mouse is the best available animal model of T1D, first developed by Makino et al \toref{Makino et 1980} in 1980.
It is a model that is genetically predisposed to develop T1D by a well characterised pathogenesis over a well defined time course as outlined below and in \toref{figure showing T1D in NOD}.

\begin{enumerate}
\item Sensitisation - the immune system first becomes aware of islet antigen and incorrectly recognises it as foreign
\item Infiltration - cells of the immune system move to the pancreatic islets and infiltrate the tissue. For a while, the environment remains regulated and no damage occurs
\item Insulitis - the immune cells in the islets become activated and cause inflammation
\item Beta cell destruction - activated cytotoxic T cells (CTL) begin to target and destroy beta cells. Beta cells are believed to die by apoptosis \citep{Cnop2005}.
\end{enumerate}

It is believed that the NOD mouse is a relevant model for T1D research as it is thought the human disease follows a similar pathogenesis.
It has been seen in donated human diabetic tissues that there is infiltration, insulitis and CTL-mediated beta cell destruction.
Each of the stages of T1D development will be discussed further in the relevant sections below.


\subsection{Initiation of T1D}

It is not known what causes T1D.
There are numerous hypotheses as to how the immune system first becomes aware of islet antigens that normally remain hidden.


One such hypothesis is that a childhood illness affecting the pancreas may cause release of islet antigen in an inflammatory environment \citep{Green1999}.
This would therefore cause activation of autoreactive T cells and allow the initiation of beta cell destruction.
However, this model for human disease would not fit with the disease seen in the NOD mouse as the highest incidence of T1D is seen in colonies housed in specific-pathogen-free facilities\citep{Delovitch1997}.
This means the mice would not have had the same virus exposure to trigger the disease and yet T1D is observed anyway.

Another possibility may involve the normal physiological process of pancreatic remodelling during growth and development.
Normally this occurs in a non-inflamed environment and as cells die and are replaced, the debris is cleared up by macrophages.
The lack of inflammatory signals means that APCs do not become activated and therefore cannot present to T cells and mount an immune response \citep{Green1999}.
However, if there is inflammatory signalling, or presence of inflammatory cytokines, the APCs can become activate and therefore able to present to T cells to ellicit an immune response.
This could suggest that the NOD mouse may be remodelling its pancreas in a more inflammatory environment, allowing the activation of APCs.
Support for this theory was provided by the production of a TNFa-NOD mouse, whereby TNFa is expressed in the islets.
The presence of this inflammatory cytokine sped up the onset of diabetes and it was seen that some beta cells apoptosed prior to DC infiltration and that the DCs became activated\citep{Green1998}.


\subsection{Infiltration and Insulitis}

Following sensistisation to islet antigen, immune cells then begin to move into the islets to form an infiltrate.
First only the edges of the islets are affected and this is known as peri-insulitis\citep{Thomas2000}.
However, soon cells move into the islets too.
Cells involved are DCs, macrophages, B cells and T cells\citep{Brodie2008}.
It is believed that insulitis is the point at which tolerance to beta cell antigen is lost \citep{Thomas2000}.
There is evidence to suggest that this pathogenesis is limited only to the NOD mouse as pancreatic samples from diabetic humans also reveal infiltrated islets.
Similarly to the NOD mouse, the infiltrate contained APCs and lymphocytes which were mainly CD8+ T cells \citep{Hanafusa2008}

\subsection{Beta cell destruction}

Following non-destructive insulitis, immune cells gain their aggressive traits and beta cells begin to be destroyed.
The destruction of beta cells in the NOD pancreatic islets is believed to be mediated by cytotoxic T lymphocytes (CTLs)\citep{Thomas2000, Brodie2008}.
Due to the large population of CD8+ lymphocytes seen in the infiltrate of diabetic pancreatic samples from human\citep{Hanafusa2008}, it is possible that these are developing into CTLs and killing off beta cells.
The destruction of beta cells in T1D leads to absolute insulin deficiency \citep{Daneman2006}



\section{Immune System Involvement}

The immune system is a complex system consisting of interlinking pathways that all work together to act as a powerful army to protect the host from bacteria, viruses and other injurious agents.
It is split broadly into two arms, the adaptive immune system, and the innate immune system, each of which will be discussed in more detail below.


\subsection{The innate immune system}
The innate immune system is the very fast acting, non specific arm of the immune system.
It is inspecific and will therefore attack any invading organisms in order to rid the body of potential threat.
It also has no immunological memory, that is, it is incapable of knowing whether or not the body has encountered the organism before and therefore cannot launch an attacked tailored to the particular invasion.

Cells of the innate immune system that are of importance to this project are macrophages and dendritic cells.
These cells are able to phagocytose antigens and present them to T cells to activate them (See below \toref{T cell activation}).

\subsubsection{Antigen Presentation}

Antigen presentation is the process by which the innate immune system can signal to the adaptive immune system that an immune response is required.
Professional APCs (macrophages, DCs and B cells (see \toref{section on B cells}) scavenge pathogens when they invade and phagocytose them.
From here they breakdown the pathogen, then express their antigen in MHC Class II.
This peptide/MHC complex can then be recognised by the T cell receptor on a T cell.
This provides signal one of T cell activation.
In inflammatory conditions, a costimulatory molecule is also expressed on the APC which provides signal two of T cell activation.
T cells are then able to carry out their functions \toref{T cell section}.

APCs are of importance in T1D as they are the cells which can present islet antigens to autoreactive T cells causing their activation.
The activated T cells are then able to destroy their target cells, beta cells.


\subsection{The adaptive immune system}

In contrast to the innate immune system, the adaptive immune system is slower-acting and specific.
The high specificity of the adaptive arm means that pathogens are recognised by cells of the adaptive immune system and therefore the immune response is designed specifically for the invading target.
The adaptive immune system also has immunological memory so it can remember what it has encountered before so that a quicker response can be mounted when the pathogen is encountered for a second time.

\subsubsection{Cells of the adaptive immune system}
The adaptive immune system plays a huge role in T1D.
T and B cells are part of this system.

\paragraph{T cells}
T cells are very important cells of the adaptive immune system.
As mentioned above, T cell activation requires the presentation of antigen by a professional APC.
Following activation, T cells are able to carry out their functions.

There are different types of T cell which have different functions as follows:
\begin{itemize}
\item CD4+ Helper T lymphocytes - CD4+ Helper T cells (CD4+ T cells) are cells which are able to help other immune cells carry out their own functions. 
Helper T cells express CD4 which allows them to respond to MHC Class II expressed on the surface of APCs. 
Activated CD4+ T cells are able to help with CD8+ cytotoxic T cell activation and B cell antibody production.
\item CD8+ Cytotoxic T lymphocytes - CD8+ cytotoxic T cells (CTL) are produced through activation of CD8+ T cells.
Once activated, CTLs are capable of killing off their target cells in a highly specific manner.
\end{itemize}

\paragraph{B cells}
B cells have two physiological roles.
Antigen presentation and Antibody production.
Whilst B cells are APCs, their main function is antibody production.
The production of antibodies relies on the help of CD4+ T helper cells as mentioned above.
The mechanism of antibody production will be described in more detail in \toref{B cell receptor development section}.
Antibodies are important for the neutralisation of toxins, opsonisation to help with phagocytosis of pathogens and destruction of bacteria and viruses.

Both B and T cells play an important role in T1D pathology


\todo{Say each cell has a role in T1D, discussed below in \toref{Involvement in T1D}}

\todo{antigen presentation and costimulation}

\section{B cells}
\subsection{Role of B cells}
B cells have two physiological roles, antigen presentation and antibody production.
Whilst B cells are APCs, their main function is antibody production.
Antibodies are an important part of the adaptive immune response which help with the neutralisation of toxins, phagocytosis of pathogens and destruction of bacteria and viruses.

B cells also play an important role in the pathogenesis of T1D. 
This will be discussed in more detail in \cref{sec:BcellsinT1D}

\subsection{Development - Stem cells to maturity}
\label{subsec:Bcelldevelopment}
\subsubsection{B cell commitment}

Haematopoietic stem cells are the progenitors from which all blood cells derive.
These cells are found in the bone marrow and, via the production of various progenitors, can produce cells of megakaryocytes, erythrocytes, lymphocytes, granulocytes, macrophages and natural killer cells.
HSCs are characterised by their Lin\textsuperscript{-} Sca\textsuperscript{high} c-kit\textsuperscript{high} expression (So called LSK cells) and lack of surface Flt3\citep{Welinder2011}.

Following the haematopoietic stem cell stage, multipotent progenitors form which have the capability of producing all types of blood cell.
These cells are also LSK but also express Flt3\textsuperscript{low}\citep{Welinder2011}.
Interestingly, MPPs can express genes of multiple lineages \citep{Hu1997} suggesting that this expression is due to priming for lineage committment, rather than commitment itself.
It is thought that by expressing these genes, it maintains them in an accessible chromatin state, ready for commitment\citep{Welinder2011}.
Evidence for this has been seen through changes in chromatin staus of lineage restricted genes during blood cell development\citep{Weishaupt2010}.

Following MPP development, magakaryocytic and erythrocytic potential is much reduced and the so-called lymphoid primed multipotent progenitor (LMPP) arises.
These cells are capable only of lymphocyte, macrophage, granulocyte and NK cell production\citep{Adolfsson2005}.
LMPPs are characterised by being LSK but also Flt3\textsuperscript{high}.

Next, Il-7Ra upregulates and Sca-1 and c-kit downregulate and to form common lymphoid progenitors (CLP).
These cells are capable only of B and T lymphocyte development and NK cell development \citep{Kondo1997} and are Sca-1\textsuperscript{low} c-kit\textsuperscript{low} Flt3\textsuperscript{+} Il-7Ra\textsuperscript{+}.
Flt3 is the receptor for Flt3 ligand (Flt3L) which is a growth promoter found in the bone marrow for CLPs.
It works synergistically with IL-7 to promote lymphocyte proliferation and this occurs early in lymphopoiesis as Flt3 acts to increase IL-7Ra expression so cells are more responsive to IL-7 \citep{Holmes2006}.

However, further research into CLPs has revealed significant heterogenity within the population.
Many different groups have used different methods, markers and reporters in order to try and identify a subpopulation of CLPs which is restricted to the B cell lineage. 
Some of these are outlined below
\begin{itemize}
\item \citet{Mansson2010} - Expression of lambda5 and Rag1 was used to divide the CLP compartment into cells that are able to produce T, B and NK cells, cells that are able to produce T and B cells, and those that are B cell restricted.
Using lambda5 reporter mice crossed with Rag1 reporter mice, it was found that CLPs could be divided into three populations.
L5-Rag1\textsuperscript{low} cells which produced T, B and NK cells, L5-Rag1\textsuperscript{high} which produced T and B cells with reduced NK cell potential, and L5\textsuperscript{+}Rag1\textsuperscript{high} cells which were B cell lineage restricted.
The also found that the L5\textsuperscript{high}Rag1\textsuperscript{high} cells had increased expression of the surface marker Ly6D.
\item \citet{Inlay2009} - Expression of Ly6D was used as a marker for determining B cell lineage commitment.
For their investigations they split CLPs into Ly6D+ and Ly6D- fractions and then looked at their ability to produce T, B and NK cells.
Interestingly, Ly6D- CLPs were able to produce all three types of progeny and were therefore termed all-lymphoid progenitors, ALPs.
On the other hand, Ly6D+ CLPs were almost totally B cell committed.
\item \citet{Zhang2013} - Expression of Ly6D and Rag1 were used as markers to follow B cell commitment and differentiation.
The CLP compartment was split based on Flt3 expression, Rag1 expression and Ly6D expression.
It was found that those cells expressing Ly6D and Rag1 were the most potent at producing B cells. 
Most of these cells expressed Flt3.
These B cell producing cells also had the highest levels of Rag1, EBF and pax5 transcripts indicating B cell lineage progression.
\end{itemize}

It therefore appears that within the CLP stage, there are subpopulations of cells which have leanings towards different lineages.
It may be that the appearance of these different subpopulation may be due to the development of the CLPs, that is, the more committed progenitors (for example, BLPs) are at a later stage of CLP development than less committed progenitors (for example, ALPs).
It appears that Rag1 and Ly6D are very important in determining B cell restricted progenitors and that it is reasonable to hypothesise that a committed B cell progenitor (BLP) could have the phenotype of Sca-1\textsuperscript{low} c-kit\textsuperscript{low} Flt3\textsuperscript{+} IL-7Ra\textsuperscript{+} Ly6D\textsuperscript{+} with evidence of previous Rag1 expression and robust levels of B cell development gene transcripts.

Following the development of B cell comitted progenitors, these cells then go on to express B220 (expressed on B cells and some plasmacytoid DCs), followed by CD19.
Originally it was thought that B cells were only committed to the B cell lineage once B220 and CD19 were being expressed, however, there is now much evidence to suggest that this committment step occurs prior to their expression.


\subsubsection{Committed B cell development}

Following commitment to the B cell lineage and the expression of CD19, B cells then progress through the following stages \citep{Cambier2007}:
\begin{itemize}
\item Pro B cell - CD19+CD43+IgM-
\item Pre B cell - CD19+CD43-IgM-
\item Immature B cell - CD19+CD43-IgM+
\end{itemize}

For progression from the pro B cell stage to the pre B cell stage, pro B cells must begin the rearrangement of their IgM heavy chain (see below).

Following the production of immature B cells in the bone marrow,they then leave the bone marrow and migrate to the spleen to mature.
This is facilitated by the increase in IgM expression during development which allows immature B cells to become transitional B cells and migrate towards the centre of the bone marrow. 
From here they are carried by the central sinus then venous circulation to the spleen \citep{Loder1999}.


\subsubsection{B cell receptor development}
\label{subsubsec:Bcellrecepdevelopment}


The formation of a B cell receptor (BcR) is vital for the development of a B cell.
The BcR may either membrane-bound or secreted in the form of antibody.

B cells are capable of recognising invading pathogens in a very specific manner thanks to it's highly specific BcR.

The B cell receptor is the B cell's way of identifying pathogens in a very specific manner.
This means that each B cell must have a different BcR in order to have repertoire capable of recognising different antigens.
In order to produce such a highly diverse repertoire, there are a few mechanisms in place during B cell development which allow the production of a highly diverse BcR across B cells.

As shown in \fig{Showing BcR structure}, the structure of a BcR consists of 2 heavy chains, which have a constant region and a variable region.
Alongside these are two light chains, which also consist of a constand and variable region.
The each light chain sits alongside the heavy chain with the two variable regions next to each other.
Between them, the variable regions make up the antigen binding site.
The constant regions give the antibody its class, for example, IgM, IgD, IgE, IgA, IgM \citep{Pieper2013}.

For both heavy and light chains, the variable region is made up of a V (variable), D (Diversity, heavy chain only) and J (joining) gene segments. 
Genomically encoded are multiple copies of different V, D and J genes and during rearrangement, one of each of these genes are selected at random and joined to produce different combinations of receptors.
This process is mediated by enzymes known as RAG recombinases which are encoded by the recombination activating genes (RAG).
These enzymes recognise recombination signal sequences (RSS) found next to the coding regions of each V, D and J gene and bring the ends of the chosen genes near to each other to be joined \citep{Fugmann2014, Oettinger1999}.
To join the segments, RAG proteins cleave at either end of the genes and leave hairpin loops of DNA.
From here a nuclease nicks the loops leaving short sequences at the end of each gene to be joined.
This allows the segments to join via non homologous end joining \citep{Schatz2011}.
Further diversity is also added via the use of an enzyme called terminal deoxynucleotidyltransferase (TdT) which adds random base pairs between each segment.
The use of TdT allows the production of the ~10\textsuperscript{14} different immunoglobulins \citep{Motea2010}.

In order to progress from the pro B to the pre B cell stage, the rearrangement of the IgM heavy chain must begin.
This has happened, it can be coupled to a surrogate light chain to form the pre-BcR which has the role of initiating light chain rearrangement so that a full IgM molecule can be produced.
The transition from pro B to pre B cells is also the first point of screening for autoreactivity\citep{Pieper2013}.

Antibodies are the secreted form of the BcR and are produced by B cells that have differentiated into plasma cells.
Antibodies are secreted into the blood stream to help with pathogen toxin neutralisation, phagocytosis and complement activation\citep{Janeway2008}.


\subsection{Bone Marrow Niche}
\subsection{Genes and Transcription Factors driving B cell development}
\subsection{Tolerance}

\section{T cells}
\subsection{Role}
\subsection{Development}
\subsection{The Thymus}
\subsection{Genes and Transcription factors}
\subsection{Tolerance}



\section{B cell involvement in T1D}
\label{sec:BcellsinT1D}
\section{Abnormal B cell presence in the NOD mouse thymus}
\subsection{Thymic B cells}


\todo{thymic B cells are increased in the NOD conpared to nondiabetic controls}

It is normal for B cells to be present in the thymus in small numbers \citep{Isaacson1987, Akashi2000} of non diabetic humans \citep{Isaacson1987} and mice \citep{Akashi2000}. 
However, in the NOD mouse, this population of B cells is dramatically increased \citep{OReilly1994}.
This information has been confirmed by previous work carried out in the laboratory where flow cytometry has been carried out on the thymi of NOD mice and non diabetic control B6 mice.
As shown in \fig{Add figure of JV data}, the presence of B cells in the NOD mouse thymi is significantly increased compared to B6 thymi and the difference between the strains increases as the mice age. (Acknowldegements to Jennifer Varian for provision of raw data). \todo{Explain the graph when it is done!}

  \citep{Christensson1998} for MG and Thymic B cells.

\subsubsection{Potential role of thymic B cells in T1D}

\todo{previous data on autoantibody production and antibody presentation by B cells in T1D}


































